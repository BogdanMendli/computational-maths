\documentclass[a4paper,11pt]{article}

\usepackage{wrapfig}
\usepackage{amsmath}
\usepackage{pgfplots}
\usepackage{enumerate}
%Russian-specific packages
%--------------------------------------
\usepackage[T2A]{fontenc}
\usepackage[utf8]{inputenc}
\usepackage[russian, english]{babel}
%--------------------------------------

\title{08. Сплайн-интерполяция. Подпрограммы \textbf{SPLINE} и \textbf{SEVAL}. Интерполирование по Эрмиту. Обратная задача интерполирования.}
\author{Андрей Бареков \and Ярослав Пылаев \and По лекциям Устинова С.М.}
\date{\today}

\begin{document}
\maketitle
\newpage

На практике интерполяционные полиномы высоких степеней строят редко, т.к. они очень чувствительны к погрешности в исходных данных. В таких случаях возможно \textit{разбиение
  исходного промежутка на ряд участков}, на каждом из которых строится полином относительно невысокой степени. \\
  На практике так поступают часто, однако в ряде приложений
  требуется дифференцируемость функции, а она отсутствует в точках сопряжения соседних полиномов. Решить проблему позволяет \textit{сплайн-интерполяция}.

\section{Сплайн-интерполяция}
Имеется $N$ точек и $N-1$ промежутков:
\[[x_1, x_2], [x_2, x_3], \dots, [x_k, x_{k+1}], \dots, [x_{N-1}, x_N]\]
На каждом промежутке $[x_k, x_{k+1}]$ строится интерполяционный полином третьей степени:
\[[x_k, x_{k+1}] \rightarrow S_k(x) = \overbrace{a_k}^{=f_k} + b_k(x-x_k) + c_k(x-x_k)^2 + d_k(x-x_k)^3\]

\begin{center}
  \large{Возникшая степень свободы используется для гладкого сопряжения соседних полиномов} \\
\end{center}

Итак, имеем \underline{\(4 \times (N - 1) = 4N - 4\) параметра}. \\

Потребуем, чтобы во всех \textit{внутренних точках} совпадали соседние полиномы, их первые и вторые производные - это и будет условием гладкого сопряжения.
\begin{equation*}
  \begin{cases}
    S_k(x_{k+1}) = S_{k+1}(x_{k+1}) \\
    S^{'}_k(x_{k+1}) = S^{'}_{k+1}(x_{k+1}) \\
    S^{''}_k{x_{k+1}} = S^{''}_{k+1}(x_{k+1})
  \end{cases},\, k = 1, 2, \dots, N - 2
\end{equation*}

\begin{center}
  Выходит \(3 \times (N - 2) = 3N - 6\) условий, $+N$ условий интерполирования (совпадение полиномов и их функции в узлах) \underline{$=4N-6$ условий}.
\end{center}
\begin{flushright}
\footnotesize{
  $N$ уравнений отражают \\
  требование интер-ния:
  $S_k(x_k)=f_k$\\
  $k=1,2,...,N-1$ (по $N$ точкам $N-1$ ст.)\\
  Обратим внимание: $S_{N-1}(x_N)=f_N$
}
\end{flushright}

\underline{Недостающие два} уравнения чаще всего задаются на концах промежутка в точках $x_1$ и $x_N$. Эти уравнения должны удовлетворять двум условиям одновременно:
\begin{enumerate}
  \item Возникшая система уравнений должна как можно проще решаться.
  \item Эти уравнения должны быть согласованы с характером поведения функции на концах промежутка.
\end{enumerate}
В предлагаемом ПО строится интерполяционный полином третьей степени $Q_3(x)$ по первым четырём точкам и приравнивается: \\
\marginpar
{
  \footnotesize {
    $d_1(x-x_k)^3$, \\
    $3d_1(x-x_k)^2$, \\
    $6d_1(x-x_k)$, \\
    $6d_1$ = const
  }
}
\begin{center}
  $S^{'''}_1(x_1) = Q^{'''}_3(x_1)$ $\Rightarrow 6d_1 = const$ \small{(дифференцировали)} \\
\end{center}
Аналогично, по последним четырём точкам строится полином $\widetilde{Q}_3(x)$ и приравнивается:
\[ S^{'''}_{N - 1}(x_N) = \widetilde{Q}^{'''}_3(x_N) \]

\section{Программное обеспечение}
ПО состоит из двух программ:
  \subsection{SPLINE}
  \[SPLINE(\underbrace{N}_{\text{Кол-во точек}},\,\underbrace{X_K}_{\text{Вектор с узлами}},\,
                    \underbrace{F_K}_{\text{Вектор с знач.}},\,\underbrace{B, C, D}_{\textbf{Выходные параметры}\ b_k, c_k, d_k})\]
  Решает систему уравнений относительно коэффициентов полиномов.
  \subsection{SEVAL}
  \[\underbrace{SEVAL}_{\textbf{function}}(N, X, X_K, F_K, \underbrace{B, C, D}_{\text{Получены из SPLINE}})\]
  Вычисляет значение сплайна в некоторой точке $X, X \in [x_k, x_{k+1}]$. \\
  $S_k(x_k)=a_k=f_k$ (поэтому ищем только $b,c,d$).\\
  Программа изначально определяет промежуток, на котором находится $X$, а потом высчитывает полином. Номер полинома находится двоичным поиском.

\section{Интерполирование по Эрмиту}
Так называется задача интерполирования, когда в таблице кроме значений функции присутствуют еще и производные. \\
\begin{minipage}{1\linewidth}
  \begin{wraptable}{l}{3cm}
    \begin{tabular}{ c|c|c|c }
      $x$   & $f(x)$ & $f^{'}(x)$ & $f^{''}(x)$ \\
      \hline
      $x_0$ & $f_0$  & $f^{'}_0$  & \\
      $x_1$ & $f_1$  & $f^{'}_1$  & $f^{''}_1$ \\
      $x_2$ & $f_2$  &            &
    \end{tabular}
  \end{wraptable}
  \vspace{5mm}
  \begin{equation*}
    \begin{cases}
      H(x_k) = f_k,\, k = 0, 1, 2 \\
      H^{'}(x_k) = f^{'}_k,\, k = 0, 1 \\
      H^{''}(x_k) = f^{''}_k,\, k = 1 \\
    \end{cases}
  \end{equation*}
\end{minipage}

Степень полинома Эрмита равна общему количеству условий $-1$. \\
Аналогично полиному Лагранжа можно записать полином Эрмита в готовом виде, не решая систему уравнений. Наиболее простой вид он имеет, когда количество условий во всех узлах одинаково.
  \subsection{Пример}
  Рассмотрим разложение функции в степенной ряд Тейлора в точке $x_0$:
  \[f(x) = \underbrace{f(x_0) + \frac{x - x_0}{1!}f^{'}(x_0) + \frac{(x - x_0)^2}{2!}f^{''}(x_0)}_{P_2(x)} + \dots\]
  Первые три слагаемых - частная сумма ряда Тейлора в $x_0$.
  \begin{equation*}
    \begin{cases}
      P_2(x_0) = f(x_0) \\
      P^{'}_2(x_0) = f^{'}(x_0) \\
      P^{''}_2(x_0) = f^{''}(x_0)
    \end{cases}
  \end{equation*}
  Так как полином $P(x)$ удовлетворяет условиям выше, то частная сумма ряда Тейлора - частный случай полинома Эрмита с одним узлом интерполирования.

\section{Обратная задача интерполирования}
\begin{minipage}{1\linewidth}
  \begin{wraptable}{l}{2.5cm}
    \begin{tabular}{ c|c }
      $x$ & $f(x)$ \\
      \hline
      $x_0$ & $f(x_0)$ \\
      $x_1$ & $f(x_1)$ \\
      $\cdots$ & $\cdots$ \\
      $x_m$ & $f(x_m)$
    \end{tabular}
  \end{wraptable}

  \begin{gather*}
    x^* \Rightarrow f(x^*) = \,? \text{ прямая задача интерполирования}\\
    \underline{f(x^*) = f^* \Rightarrow x^* = \,? \text{ восстановить аргумент}}
  \end{gather*}
\end{minipage}
\vspace{1cm}
\begin{enumerate}[I.]
  \item \textit{Строим интерполяционный полином}, приравниваем значение функции, ищем корни: $Q_m(x) = f^*$. Возникшее уравнение решается либо аналитически,
        либо приближённо численными методами.
  \item \textit{Меняем местами столбцы таблицы}, строим интерполяционный полином для обратной функции, и вычисляем значение этого полинома в точке $f^*$ и находим $x^*$. \\
  \textbf{Ограничение}: Для существования обратной функции, исходная должна быть строго монотонной. В противном случае делим на промежутки монотонности.
\end{enumerate}

\end{document}

\documentclass[a4paper,11pt]{article}

\usepackage{wrapfig}
\usepackage{amsmath}
\usepackage{pgfplots}
\usepackage{enumerate}
\usepackage{cancel}
\usepackage[most]{tcolorbox} % для управления цветом
%Russian-specific packages
%--------------------------------------
\usepackage[T2A]{fontenc}
\usepackage[utf8]{inputenc}
\usepackage[russian, english]{babel}
%--------------------------------------

%Для выделения блока текста в рамку
\definecolor{block-gray}{gray}{0.95} % уровень прозрачности (1 - максимум)
\newtcolorbox{importantblock}{colback=block-gray,grow to right by=-10mm,grow to left by=-10mm,
boxrule=0pt,boxsep=0pt,breakable} % настройки области с изменённым фоном

\definecolor{lemonchiffon}{rgb}{1.0, 0.98, 0.8}
\newtcolorbox{mainblock}{colback=lemonchiffon,grow to right by=-10mm,grow to left by=-10mm,
boxrule=0pt,boxsep=0pt,breakable} % настройки области с изменённым фоном

\makeatletter
\newcommand{\settag}[1]{
  \tag*{(#1)\qquad}
  \edef\@currentlabel{\theequation#1}}
\makeatother

\title{22. Методы бисекции, секущих, обратной параболической интерполяции для решения
       нелинейных уравнений. Подпрограмма ZEROIN}
\author{Андрей Бареков \and Ярослав Пылаев \and По лекциям Устинова С.М.}
\date{\today}

\begin{document}
\maketitle
\newpage

\section{Решение нелинейных уравнений}
Остановимся на уравнении
\begin{equation}
  f(x) = 0.
  \label{eq:NonlinearEq}
\end{equation}
Решению уравнения (\ref{eq:NonlinearEq}) предшествует подготовительный этап выявления промежутка $[a, b]$, на котором только один $0$
и $f(a)f(b) < 0$. Так будет найден единственный вещественнй корень уравнения.

\subsection{Метод бисекции (или метод \textit{дихотомии}, или метод \textit{половинного деления})}
\begin{center}
  \begin{tikzpicture}
    \begin{axis} [
      xlabel = {}, ylabel = {},
      xmin = 0, xmax = 8, ymin = = -3, ymax = 5,
      xtick = {}, ytick = {},
      no markers,
      extra x ticks={1, 3, 7},	% положение дополнительных тиков
      extra x tick labels={	% их подписи
 		   {\large{$\textbf{a}$}}, {\large{$\textbf{c}$}}, {$\textbf{d}$}
      },
    ]
      \addplot [
        color = black,
        thin,
      ] coordinates{ (0,0)(8,0)};
      \addplot [
        color = blue,
        thick,
        smooth,
        domain = 0:7,
        samples = 300,
      ] coordinates { (0,6)(1,3)(1.5,1.9)(2,1.1)(3,0)(4,-0.8)(5,-1.3)(6,-1.5)(7,-1.58)(8,-1.6)(10,-1.65)};

      \addplot [
        color = red,
        only marks,
      ] coordinates {(1,3.05)};
      \addplot [
        color = green,
        only marks,
      ] coordinates {(3,0)};
      \addplot [
        color = red,
        only marks,
      ] coordinates {(7,-1.58)};
    \end{axis}
  \end{tikzpicture}
\end{center}
Вычисляем значение функции в точке \[c = \frac{a+b}{2}\] и вычисляем $f(c)$. Если $sign(f(a)) = sign(f(c))$, то "отбрасываем" эту половину, иначе "отбрасываем" вторую половину. Далее рекурсия...\\

\noindent За одно вычисление $f(x)$ на шаге промежуток гарантированно сокращается \textit{в два раза} независимо от вида функции.
Если поведение функции хорошо прогнозируется и она имеет простой вид, то ответ можно получить и быстрее.

\subsection{Метод секущих}
По двум точкам строим интерполяционный полином Лагранжа первой степени и очередное приближение - $0$ этого полинома
\begin{gather*}
  Q_1(x) = \frac{x-b}{a-b}f(a) + \frac{x-a}{b-a}f(b), \\
  c = a - \frac{b-a}{f(b)-f(a)}f(a).
\end{gather*}
Метод секущих с разделенной разностью выглядит следующим образом:
\begin{equation*}
  x_{n+1} = x_n - \frac{x_n-x_{n-1}}{f(x_n)-f(x_{n-1})}f(x_n).
\end{equation*}
Новый промежуток будет $[c, b]$ или $[a, c]$ в зависимости от знака $f(x)$ в точке $c$. Замедление сходимости метода часто происходит,
когда очередное приближение получается слишком близко к одному из концов промежутка.

\subsection{Метод обратной параболической интерполяции}
Если функция вычислена более, чем в двух точках, то эту информацию тоже можно использовать. \\

\noindent По трем точкам $x_k,x_{k-1},x_{k-2}$ строится интерполяционнй полином второй степени для \textit{обратной} функции,
при этом выполянются условия \\$x_i=g(f_i), \hspace{1mm} i =k, k-1, k-2$. В качестве следующего приближения берется $x_{k+1}=g(0)$.
Одна из предыдущих точек удаляется. \\

\noindent Важно, чтобы три значени $f_i$ были бы различными, чтобы избежать деления на $0$ (см. формулы "Метода секущих").

\section{Подпрограмма ZEROIN}
\begin{flushright}
  \textit{Всегда работает хорошо.}
\end{flushright}
\begin{equation*}
  \underbrace{ZEROIN}_{function}(A, B, \underbrace{F}_{f(x)}, EPS).
\end{equation*}
Внутри программы два алгритма:
\begin{itemize}
  \item метод обратной параболической интерполяции (основной),
  \item метод бисекции (используется на несколько шагов, если основной метод замедляется).
\end{itemize}

\end{document}

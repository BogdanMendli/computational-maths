\documentclass[a4paper,11pt]{article}

\usepackage{wrapfig}
\usepackage{amsmath}
\usepackage{pgfplots}
%Russian-specific packages
%--------------------------------------
\usepackage[T2A]{fontenc}
\usepackage[utf8]{inputenc}
\usepackage[russian, english]{babel}
%--------------------------------------

\title{04. Разделённые разности и их связь с конечными.}
\author{Андрей Бареков \and Ярослав Пылаев \and По лекциям Устинова С.М.}
\date{\today}

\begin{document}
\maketitle
\newpage

\section{Разделённые разности}

Для равноотстоящих узлов таблицы конечные разности являются хорошей характеристикой изменения функции, аналогичной производной для непрерывного случая.
При произвольном же расположении узлов таблицы целесообразно ввести понятие \textit{разделенной разности}.

\[f(x_k; x_{k+1}) = \frac{f_{k+1} - f_k}{x_{k+1} - x_k}\] \\

\textbf{Аналогично производным высших порядков вводятся разделённые разности высших порядков:}

\begin{gather*}
  f(x_k; x_{k+1}; x_{k+2}) = \frac{f(x_{k+1}; x_{k+2}) - f(x_k; x_{k+1})}{x_{k+2} - x_k} \\
  \dots \\
  f(x_k; x_{k+1}; \dots; x_{k+s}) = \frac{f(x_{k+1}; x_{k+2}; \dots; x_{k+s})
      - f(x_k; x_{k+1}; \dots; x_{k+s-1})}{x_{k+s} - x_k}
\end{gather*}

\textbf{Аналогично вводятся конечные разности {\textit{(П.1)}} высших порядков:}

\begin{gather*}
  \Delta f_k = f_{k+1} - f_k \\
  \Delta^2 f_k = \Delta f_{k+1} - \Delta f_k = f_{k+2} - f_{k+1} - f_{k+1} + f_k = f_{k+2} - 2f_{k+1} + f_k \\
  \Delta^3 f_k = \Delta^2 f_{k+1} - \Delta^2 f_k = f_{k+3} - 3f_{k+2} + 3f_{k+1} - f_k \\
  \dots
\end{gather*}

Если узлы таблицы равноотстоящие, то можно использовать и конечные, и разделённые разности.

  \subsection{Связь между конечными и разделенными разностями для равноотстоящих узлов}

  \begin{gather*}
    f(x_k; x_{k+1}) = \frac{f_{k+1} - f_k}{x_{k+1} - x_k} = \frac{\Delta f_k}{h} \\
    f(x_k; x_{k+1}; x_{k+2}) = \frac{f(x_{k+1}; x_{k+2}) - f(x_k; x_{k+1})}{x_{k+2} - x_k}
        = \frac{\frac{\Delta f_{k+1}}{h} - \frac{\Delta f_k}{h}}{2h} = \frac{\Delta^2 f_k}{2h^2}
  \end{gather*}
  По индукции легко показать, что: \\
  \[f(x_k; x_{k+1}; \dots; x_{k+s}) = \frac{\Delta^s f_k}{s! \times h^s}\]
\newpage

\end{document}

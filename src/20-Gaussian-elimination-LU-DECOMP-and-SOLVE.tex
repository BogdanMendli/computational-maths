\documentclass[a4paper,11pt]{article}

\usepackage{wrapfig}
\usepackage{amsmath}
\usepackage{pgfplots}
\usepackage{enumerate}
\usepackage{cancel}
\usepackage[most]{tcolorbox} % для управления цветом
%Russian-specific packages
%--------------------------------------
\usepackage[T2A]{fontenc}
\usepackage[utf8]{inputenc}
\usepackage[russian, english]{babel}
%--------------------------------------

%Для выделения блока текста в рамку
\definecolor{block-gray}{gray}{0.95} % уровень прозрачности (1 - максимум)
\newtcolorbox{importantblock}{colback=block-gray,grow to right by=-10mm,grow to left by=-10mm,
boxrule=0pt,boxsep=0pt,breakable} % настройки области с изменённым фоном

\definecolor{lemonchiffon}{rgb}{1.0, 0.98, 0.8}
\newtcolorbox{mainblock}{colback=lemonchiffon,grow to right by=-10mm,grow to left by=-10mm,
boxrule=0pt,boxsep=0pt,breakable} % настройки области с изменённым фоном

\makeatletter
\newcommand{\settag}[1]{
  \tag*{(#1)\qquad}
  \edef\@currentlabel{\theequation#1}}
\makeatother

\title{20. Метод Гаусса и явление плохой обусловленности. LU-разложение матрицы. Подпрограммы DECOMP и SOLVE}
\author{Андрей Бареков \and Ярослав Пылаев \and По лекциям Устинова С.М.}
\date{\today}

\begin{document}
\maketitle
\newpage

\section{Явление плохой обусловленности матриц}
Дана линейная алгебраическая система:
\begin{equation}
  Ax=b, \hspace{5mm} det(A) \ne 0,
  \label{eq:Eq}
\end{equation}
\begin{align*}
  A &- \text{квадртная матрица}, \\
  b &- \text{заданный вектор-столбец}, \\
  x &- \text{искомый вектор-столбец}.
\end{align*}
Как погрешность в исходных данных влияет на точность решения?
\begin{importantblock}
  Если малые изменения в исходной информации приводят к сильному изменению решения, то такая матрица (и сама система (\ref{eq:Eq})) называется \textit{плохо обусловленной}.
\end{importantblock}
\vspace{3mm}
\noindent Получим \textit{количественные характеристики} плохой обусловленности.
\begin{enumerate}[I.]
  \item
    \begin{gather*}
      b \rightarrow b + \varDelta b \Rightarrow x \rightarrow x + \varDelta x, \\
      A(x+\varDelta x) = b+\varDelta b, \hspace{3mm}
      \cancel{Ax}+A\varDelta x = \cancel{b}+\varDelta b, \\
      A\varDelta x = \varDelta b, \hspace{3mm}
      \varDelta x = A^{-1}\varDelta b, \\
      \parallel\varDelta x\parallel \le \parallel A^{-1}\parallel \cdot \parallel\varDelta b\parallel, \hspace{3mm}
      \parallel b\parallel \le \parallel A\parallel \cdot \parallel x\parallel, \\
      \text{перемножим эти два неравенства и разделим результат на $\parallel x\parallel\cdot\parallel b\parallel$}: \\
      \frac{\parallel\varDelta x\parallel}{\parallel x\parallel} \le \underbrace{\parallel A\parallel \cdot \parallel A^{-1}\parallel}
        _{cond(A)} \cdot \frac{\parallel\varDelta b\parallel}{\parallel b\parallel}.
    \end{gather*}
    \begin{flushright}
      \footnotesize {
      $\frac{\parallel\varDelta x\parallel}{\parallel x\parallel}$ - относительная погрешность результата, \\
      $\frac{\parallel\varDelta b\parallel}{\parallel b\parallel}$ - относительная погрешность исходных данных.}
    \end{flushright}
    $cond(A)$ - \textit{число обусловленности матрицы} или коэффициент усиления погрешности.
    Если $cond(A) \gg 1$, то система - плохо обусловлена.
  \item
    \begin{gather*}
      A \rightarrow A+\varDelta A \Rightarrow x \rightarrow x+\varDelta x, \\
      (A+\varDelta A)(x+\varDelta x) = b, \\
      \cancel{Ax}+A\varDelta x + \varDelta A(x+\varDelta x) = \cancel{b}, \\
      A^{-1} \bigg|\hspace{3mm} A\varDelta x = - \varDelta A(x+\varDelta x), \\
      \varDelta x = A^{-1}\varDelta A(x+\varDelta x), \\
      \text{берем норму и делим на $\parallel x+\varDelta x\parallel$}: \\
      \frac{\parallel\varDelta x\parallel}{\parallel x+\varDelta x\parallel} \le \parallel A^{-1}\parallel \cdot \parallel\varDelta A\parallel \cdot
        \frac{\parallel A\parallel}{\parallel A\parallel}, \\
      \frac{\parallel\varDelta x\parallel}{\parallel x+\varDelta x\parallel} \le  \underbrace{\parallel A\parallel \parallel \cdot A^{-1}\parallel}_{cond(A)}
        \cdot \frac{\parallel\varDelta A\parallel}{\parallel A\parallel}.
    \end{gather*}
\end{enumerate}
Пример: \\
\begin{align*}
  &10^{-5} \le \underline{10^7} \cdot 10^{-12} - \text{погрешнось размерной сетки}, \\
  &10 \le \underline{10^4} \cdot 10^{-3} - \text{инженерная погрешность}
\end{align*}
Таким образом, все зависит от задачи.\\

Существуют и другие числа обусловленности, так плохо обусловленной мтарицей является матрица с большим разбросом собственных значений:
\begin{equation*}
  k(A) = \frac{|\lambda_k|_{max}}{|\lambda_k|_{min}}.
\end{equation*}
Легко показать, что $k(A) \le cond(A)$:
\begin{gather*}
  \frac{|\lambda_k|_{max}}{|\lambda_k|_{min}} \le \parallel A\parallel \parallel \cdot A^{-1}\parallel, \\
  \text{по Следствию 2 к Теореме 5 } |\lambda_k| \le \parallel A\parallel, \frac{1}{|\lambda_k|} \ge \parallel A^{-1}\parallel.
\end{gather*}

Покажем, что матрица с большим разбросом собственных значений является плохо обусловленной, что приводит к большой погрешности.\\

\noindent \textit{Утверждение 1.} Максимальные по модулю элементы матрицы $A$ имеют величину порядка $|\lambda_k|_{max}$, а могут и значительно ее превышать.
Это следует из неравенства $\lambda_k| \le \parallel A\parallel$. \\

\noindent \textit{Утверждение 2.} Собственные значения $\lambda_k$ могут изменяться на величину порядка элементов матрицы возмущения $\varDelta A$.
\begin{gather*}
  \varDelta A = \varepsilon E, \hspace{3mm} A+\varDelta A \rightarrow A+\varepsilon E, \\
  A \rightarrow \lambda_k, \hspace{3mm} A+\varepsilon E \rightarrow \lambda_k+\varepsilon.
\end{gather*}
Пусть относительная погрешность элементов матрицы $A$ $\delta \ll 1$. \\

\textit{Шаг 1.}
\begin{gather*}
  \varDelta A \rightarrow \delta|a_{ik}|_{max}, \\
  \text{по утверждению 1: } \varDelta A \rightarrow \delta|\lambda_k|_{max}.
\end{gather*}

\textit{Шаг 2.} В соответствии с утверждением 2 $\lambda_k\pm\delta|\lambda_k|_{max}$.
  $|\lambda_k|_{max}$ изменится крайне незначительно, а $|\lambda_k|_{min}$ может измениться достаточно сильно. \\

\textit{Шаг 3.}
  \begin{gather*}
    A \rightarrow \lambda_k, A^{-1} \rightarrow \frac{1}{\lambda_k}, \\
    \frac{1}{|\lambda_k|_{min}} - \text{самые большие собственные значения } A^{-1}.
  \end{gather*}
  Если $|\lambda_k|_{min}$ изменилось сильно, то сильно изменились элементы обратной матрицы, а, следовательно, сильно изменилось решение.
  \begin{importantblock}
    Врем в $min$ собственном значении, значит врем в обратной матрице.
  \end{importantblock}
\begin{gather*}
  Ax=b, \\
  x=A^{-1}b,
\end{gather*}

Описанные неприятности будут появляться тем быстрее, чем \textit{больше разброс} элементов матрицы $A$.

\section{Метод Гаусса}
Методы решения системы (\ref{eq:Eq}) делятся на две большие группы:
\begin{itemize}
  \item точные,
  \item итерационные.
\end{itemize}
\textit{Точные} методы в отсутствии ошибок округления позволяют получить точные решения за конечное число арифметических операций. \\
В ходе применения \textit{итерационных} методов рождается последовательность векторов, сходящихся к решению. \\

Типичный точный метод - \textbf{метод Гаусса}. Его "грубая" схема выглядит следующим образом: \\

\noindent \textit{Прямой ход.} В первом уравнении делим все на $a_{11}$, выражем $x_1$, подставляем во все остальные уравнения и приводим подобные. \\
Аналогично из второго уравнения \textit{исключаем} $x_2$ и т.д. В результате получаем верхнюю треугольную матрицу.\\

\noindent \textit{Обратный ход.}
\begin{equation*}
  \begin{pmatrix}
    * & * & * \\
      & * & * \\
    0 &   & *
  \end{pmatrix}
\end{equation*}
Из последнего уравнения находим $x_N$, из прелпослднего - $x_{N-1}$, и т.д.\\

Недостаток "грубой" схемы - возможное деление на нулевой или близкий к нулю элемент. Для его устранения используется \textit{выбор ведущего элемента}. \\

\noindent \textit{Вариант 1.} На $k$-м шаге в оставшейся матрице находят самый большой по модулю элемент. Затем строки и столбцы меняют так, чтобы этот элемент поменялся местами с $a_{kk}$,
\begin{equation*}
  \begin{pmatrix}
    . & . & . & . & . \\
    0 & . & . & . & . \\
    0 & 0 & a_{kk} & * & * \\
    0 & 0 & * & * & * \\
    0 & 0 & * & * & *
  \end{pmatrix}.
\end{equation*}
Делим уже на него и далее по алгоритму. \textit{Но} в этом варианте при перестановке столбцов меняется нумерация компонент вектора $x$. \\

\noindent \textit{Вариант 2.} На $k$-м шаге максимальный по модулю элемент разыскивают только в $k$-м столбце, в таком случае переставляются только строки. \\
На практике этот вариант используют чаще.

\subsection{Требования к хорошей программе метода Гаусса}
\begin{enumerate}
  \item Должен быть реализован выбор ведущего элемента.
  \item Программа должна оценивать число обусловленности матрицы - $cond(A)$.
  \item Программа должна эффективно решать несколько систем уравнений с одной и той же матрицей $A$ и различными векторами $b$.
\end{enumerate}
Реализовать 3-е требование помогает $LU$-разложение матриц.

\section{$LU$-разложение матрицы}
$LU$-разложение матрицы - представление матрицы в следующем виде:
\begin{gather*}
  A = LU, \text{где} \\
  L - \text{левая треугольная матрица с единичной диагональю}, \\
  U - \text{правая треугольная матрица, на диагонали - все, что угодно}.
\end{gather*}
Если $LU$-разложение построено, то решение исходной системы $Ax=b$ сводится к решению двух систем с трегольными матрицами:
\marginpar
{
  \vspace{11mm}
  \footnotesize ввели обознач.
}
\begin{gather}
  Ly=b, \\ Ux=y.
\end{gather}

$LU$-разложение выполняется \text{однократно}, а системы (2) и (3) решаются столько раз, сколько имеется различных векторов $x$. \\

\subsection{Пример - матрица $4\times 4$}
\begin{equation*}
  A = A_0 =
  \begin{pmatrix}
    a_{11} & a_{12} & a_{33} & a_{14} \\
    a_{21} & a_{22} & a_{23} & a_{24} \\
    a_{31} & a_{32} & a_{33} & a_{34} \\
    a_{41} & a_{42} & a_{43} & a_{44} \\
  \end{pmatrix},
\end{equation*}
Рассмотрим левую треугольную матрицу $M_1$, которая отличается от единичной первым столбцом:
\begin{equation*}
  M_1 =
  \begin{pmatrix}
    1 & 0 & 0 & 0 \\
    -a_{21}/a_{11} & 1 & 0 & 0 \\
    -a_{31}/a_{11} & 0 & 1 & 0 \\
    -a_{41}/a_{11} & 0 & 0 & 1
  \end{pmatrix},
\end{equation*}
Умножим исходную матрицу на матрицу $M_1$ слева:
\begin{equation*}
  A_1 = M_1 A_0 =
  \begin{pmatrix}
    a_{11} & \dots & \dots & a_{14}^* \\
    0 & \hdotsfor{3} \\
    0 & \hdotsfor{3} \\
    0 & \hdotsfor{3} \\
  \end{pmatrix}.
\end{equation*}
На втором шаге строим матрицу $M_2$, которая отличается от единичной вторым столбцлом ($^*$ - элементы матрицы $A_1$):
\begin{equation*}
  M_2 =
  \begin{pmatrix}
    1 & 0 & 0 & 0 \\
    0 & 1 & 0 & 0 \\
    0 & -a_{32}^*/a_{22}^* & 1 & 0 \\
    0 & -a_{42}^*/a_{22}^* & 0 & 1
  \end{pmatrix},
\end{equation*}
\begin{gather*}
  A_2 = M_2A_1 =
  \begin{pmatrix}
    + & + & \dots & a_{14}^{**} \\
    0 & + & \hdotsfor{2} \\
    0 & 0 & \hdotsfor{2} \\
    0 & 0 & \hdotsfor{2}
  \end{pmatrix}, \\
  + - \text{элементы не обязательно нулевые}.
\end{gather*}
На последнем шаге умножаем $A_2$ на $M_3$, где
\begin{equation*}
  M_3 =
  \begin{pmatrix}
    1 & 0 & 0 & 0 \\
    0 & 1 & 0 & 0 \\
    0 & 0 & 1 & 0 \\
    0 & 0 & -a_{43}^{**}/a_{33}^{**} & 1
  \end{pmatrix}.
\end{equation*}
Тогда
\begin{gather*}
  A_3 = M_3A_2 = \underbrace{M_1M_2M_3}_{M}A = U, \\
  M^{-1} \bigg| \hspace{3mm} MA=U, \\
  A = M^{-1}U, \hspace{3mm} M^{-1} = L, \\
  A = LU.
\end{gather*}
\begin{flushright}
  \footnotesize Здесь учитывались "Задачи на матрицы" 6 и 7.
\end{flushright}

\section{Подпрограммы DECOMP и SOLVE}
Программное обеспечение состоит из двух программ:
\begin{itemize}
  \item $DECOMP(NDIM, N, A, COND, IPVT, WORK)$ \\
    Выполянет $LU$ разложение матрицы, \vspace{2mm} \\
    $NDIM$ - длина столбца матрицы $A$ в операторе описания, \\
    $N$ - порядок матрицы (системы уравнений), \\
    $A$ - исходная матрица, в ней же будут размещены матрицы $L$ и $U$, \\
    $COND$ - оценка числа обусловленности, \textit{выходной}\\
    $IPVT$ - вектор индексов ведущих элементов, его $k$-я компонента указывает, какое уравнение используется для исключения $x_k$ на \\ $k$-м шаге, \\
    $WORK$ - одномерный рабочий массив (размерность $N$).
  \item $SOLVE(NDIM, N, A, B, IPVT)$ \\
    Решает системы ($2$) и ($3$) с треугольными матрицами, \\
    $B$ - вектор решения $x$, сначала в нем правые части системы ($1$).
\end{itemize}

\end{document}

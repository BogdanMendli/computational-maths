\documentclass[a4paper,11pt]{article}

\usepackage{wrapfig}
\usepackage{amsmath}
\usepackage{pgfplots}
\usepackage[most]{tcolorbox} % для управления цветом
%Russian-specific packages
%--------------------------------------
\usepackage[T2A]{fontenc}
\usepackage[utf8]{inputenc}
\usepackage[russian, english]{babel}
%--------------------------------------

%Для выделения блока текста в рамку
\definecolor{block-gray}{gray}{0.95} % уровень прозрачности (1 - максимум)
\newtcolorbox{importantblock}{colback=block-gray,grow to right by=-10mm,grow to left by=-10mm,
boxrule=0pt,boxsep=0pt,breakable} % настройки области с изменённым фоном

\makeatletter
\newcommand{\settag}[1]{
  \tag*{(#1)\qquad}
  \edef\@currentlabel{\theequation#1}}
\makeatother

\title{12. Задача численного дифференцирования. Влияние вычислительной погрешности}
\author{Андрей Бареков \and Ярослав Пылаев \and По лекциям Устинова С.М.}
\date{\today}

\begin{document}
\maketitle
\newpage

\section{Задача численного дифференцирования}
\begin{minipage}{1\linewidth}
  \begin{wraptable}{l}{4.5cm}
    \begin{tabular}{ c|c|c }
      $x$ & $f(x)$ & $f^{'}(x)$ \\
      \hline
      $x_0$ & $f(x_0)$ & $?$ \\
      $x_1$ & $f(x_1)$ & $?$ \\
      $\cdots$ & $\cdots$ & $\cdots$ \\
      $x_m$ & $f(x_m)$ & $?$ \\
    \end{tabular}
  \end{wraptable}
  Для таблично заданной функции требуется оценить значения производной $f(x)$ в узлах таблицы.
  \begin{center}
    Но всегда помним, что писал \\
    Хемминг: \textit{"Прежде, чем решать задачу, подумай, что делать с ее решением."}
  \end{center}
\end{minipage}

\vspace{5mm}
\begin{center}
  PLOT THERE
\end{center}
Для функции $f(x)$ аппроскимирующая функция $g(x)$ достаточно близка, поэтому близки и их интегралы,
но производные совпадают лишь в нескольких точках.\\

Рассмотрим еще один убедительный пример. Имеются две функции $f(x)$ и $g(x)$ и их производные:
\begin{align*}
  f(x) &; & &g(x)=f(x)+\frac{1}{N}sin(N^2x); \\
  \frac{df(x)}{dx} &; & &\frac{dg(x)}{dx}=\frac{df(x)}{dx}+Ncos(N^2x).
\end{align*}
Очевидно, что, чем больше N,тем функции становятся ближе друг к другу, но в тот же момент их производные - наоборот.
Таким образом, \textit{близость f(x) и g(x) еще не гарантирует близости их производных.} \\

Для численного дифференцирования:
\begin{itemize}
  \item функция должна изменяться достаточно плавно,
  \item шаг должен быть согласован с быстротой изменения функции,
  \item таблица не должна быть "зашумлена" погрешностью исходных данных.
\end{itemize}

Аппроксимуруем исходную функцию интерполяционным полиномом, и его производная
даст формулу численного дифференцирования, а производная от остаточного члена позволит оценить погрешность:
\begin{importantblock}
  \begin{gather*}
    f(x)=Q_m(x)+R_m(x) \\
    f^{'}(x_k) \approx Q_m^{'}(x_k) \settag{*} \\
    \varepsilon = R_m^{'}(x_k) \settag{**} \\
  \end{gather*}
\end{importantblock}

Получим формулы для численного дифференцирования для равноотстоящих узлов:
\begin{equation*}
  Q_1(x)=\frac{x-x_{k+1}}{x_k-x_{k+1}}f_k + \frac{x-x_k}{x_{k+1}-x_k}f_{k+1}
\end{equation*}
Дифференцируем и получаем значение производной в точках $x_k$ и $x_{k+1}$
\begin{gather}
  f^{'}(x_k) \approx \frac{f_{k+1}-f_k}{h} \\
  f^{'}(x_{k+1}) \approx \frac{f_{k+1}-f_k}{h}
\end{gather}
\begin{equation*}
  \varepsilon_1 = R_m^{'}(x_k)
\end{equation*}
\marginpar
{
  \vspace{2mm}
  \footnotesize{$\eta \rightarrow \eta(x)$}
}
\begin{equation*}
  R_1(x) = \frac{(x-x_k)(x-x_{k+1})}{2!}f^{''}(\eta)
\end{equation*}
\begin{center}
  \small{Остаточный член интерполяционного полинома}
\end{center}
При взятии производной получаются три слагаемых
\begin{equation*}
  R_1^{'}=\frac{x-x_{k+1}}{2!}f^{''}(\eta) + \frac{x-x_k}{2!}f^{''}(\eta) + \frac{(x-x_k)(x-x_{k+1})}{2!}f^{'''}(\eta)\eta^{'}(x),
\end{equation*}
два из которых обращаются в 0 при подстановке $x_k$ и $x_{k-1}$:
\begin{gather*}
  \varepsilon_1(x_k)= - \frac{h}{2}f^{''}(\eta) \settag{1*} \\
  \varepsilon_1(x_{k+1})=\frac{h}{2}f^{''}(\eta) \settag{2*}.
\end{gather*}

Сделаем аналогичные вычисления для интерполяционного полинома 2-й стпени $Q_2(x)$ с узлами $x_k, x_{k+1},x_{k+2}$.
\begin{gather}
  f^{'}(x_k) \approx \frac{1}{2h}(-3f_k+4f_{k+1}-f_{k+2}) \\
  f^{'}(x_{k+1}) \approx \frac{1}{2h}(f_{k+2}-f_k) \\
  f^{'}(x_{k+2}) \approx \frac{1}{2h}(3f_{k+2}-4f_{k+1}+f_k)
\end{gather}
Погрешность
\begin{equation*}
  R_2(x)=\frac{(x-x_k)(x-x_{k+1})(x-x_{k+2})}{3!}f^{'''}(\eta)
\end{equation*}
\begin{gather*}
  \varepsilon_2(x_k) = \frac{h^2}{3}f^{'''}(\eta) \settag{3*} \\
  \varepsilon_2(x_{k+1}) = - \frac{h^2}{6}f^{'''}(\eta) \settag{4*} \\
  \varepsilon_2(x_{k+2}) = \frac{h^2}{3}f^{'''}(\eta) \settag{5*}
\end{gather*}

\begin{importantblock}
  На меньшую погрешность можно расчитывать при использовании формулы (4), что и делает ее наиболее популярной;
  формулы (3) и (5) используются для дифференцирования в начале и конце таблицы.\\
\end{importantblock}

Продифференцируем интерполяционный полином 2-й степени дважды, тогда простейшая формула для 2-й производной:
\begin{equation}
  f^{''}(x_{k+1}) \approx \frac{f_{k+2}-2f_{k+1}+f_k}{h^2}
\end{equation}
Привлекая новые узлы интерполирования, можно так и дальше дифференцировать полиномы.\\

Важным на практике является \textit{выбор шага $h$}. Ограничение сверху накладывается величиной погрешности,
а снизу - точностью табличных данных для $f(x)$.

\section{Влияние вычислительной погрешности}
\marginpar
{
  \vspace{6mm}
  \footnotesize{это формулы (1) и (1*)}
}
\begin{gather*}
  f^{'} \approx \frac{f_{k+1}-f_k}{h} \\
  \varepsilon_1 = - \frac{h}{2}f^{''}(\eta)
\end{gather*}
Если значения $f_k$ и $f_{k+1}$ определены с погрешностью $\Delta_k$ и $\Delta_{k+1}$ соответсвенно, то
\begin{equation*}
  \boxed{|\varepsilon_{\text{общ}}| \le \frac{h}{2}|f^{''}(\eta)| + \frac{|\Delta_{k+1}-\Delta_K|}{h}}
\end{equation*}

Оптимальное значение шага $h_{\text{опт}}$ отвечает ситуации, когда оба слагаемых равны друг другу.
\begin{center}
  PLOT HERE
\end{center}

\end{document}

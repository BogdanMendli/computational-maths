\documentclass[a4paper,11pt]{article}

\usepackage{wrapfig}
\usepackage{amsmath}
\usepackage{pgfplots}
\usepackage[most]{tcolorbox} % для управления цветом
\usepackage{chngcntr}
%Russian-specific packages
%--------------------------------------
\usepackage[T2A]{fontenc}
\usepackage[utf8]{inputenc}
\usepackage[russian, english]{babel}
%--------------------------------------

%Для выделения блока текста в рамку
\definecolor{block-gray}{gray}{0.95} % уровень прозрачности (1 - максимум)
\newtcolorbox{importantblock}{colback=block-gray,grow to right by=-10mm,grow to left by=-10mm,
boxrule=0pt,boxsep=0pt,breakable} % настройки области с изменённым фоном

\counterwithin*{equation}{section}% Сбрасывает счетчик формул в \section

\title{06. Интерполяционный полином Лагранжа. Остаточный член полинома Лагранжа.}
\author{Андрей Бареков \and Ярослав Пылаев \and По лекциям Устинова С.М.}
\date{\today}

\begin{document}
\maketitle
\newpage

\section{Интерполяционный полином Лагранжа}
\begin{equation}
  Q_m(x_i) = f(x_i), \, i = 0, 1, \dots, m
  \label{eq:PolSys}
\end{equation}
Построить интерполяционный полином можно и в явном виде, не решая систему (\ref{eq:PolSys}).
\begin{gather*}
  \omega(x) = (x - x_0)(x - x_1)\dots(x - x_m) \\
  \omega_k(x) = \frac{\omega(x)}{x - x_k} = (x - x_0)\dots(x - x_{k-1})(x - x_{k+1})\dots(x - x_m)
\end{gather*}

\begin{equation}
  Q_m(x) = \sum_{k=0}^{m} \frac{\omega_k(x)}{\omega_k(x_k)} f(x_k)
  \label{eq:LgPol}
\end{equation}
Уравнение (\ref{eq:LgPol}) - интерполяционный полином (в форме) Лагранжа. \\
По построению очевидно, что $Q_m(x)$ - полином степени $m$.
\[Q_m(x_i) = \frac{\omega_i(x_i)}{\omega_i(x_i)} f(x_i) = f(x_i), \, i=0,1,\dots,m\]
\(\omega_k(x_i) = 0\), если \(k \ne i\)

  \subsection{Алгоритм построения полинома Лагранжа}
  \begin{enumerate}
    \item Числитель с пропущенной точкой.
    \item Знаменатель - числитель, где вместо $x$ подставляем пропущенную точку.
    \item Значение функции в пропущенной точке.
  \end{enumerate}
  \textbf{Пример для полинома $Q_2$:}
  \[Q_0(x) = f(x_0)\]
  \[Q_1(x) = \frac{x-x_1}{x_0-x_1}f(x_0) + \frac{x-x_0}{x_1-x_0}f(x_1)\]
  \[Q_2(x) = \frac{(x-x_1)(x-x_2)}{(x_0-x_1)(x_0-x_2)}f(x_0) + \frac{(x-x_0)(x-x_2)}{(x_1-x_0)(x_1-x_2)}f(x_1) + \frac{(x-x_0)(x-x_1)}{(x_2-x_0)(x_2-x_1)}f(x_2)\]

\section{Остаточный член интерполяционного полинома Лагранжа}
\begin{equation}
  f(x) = Q_m(x) + R_m(x)
\end{equation}
\(R_m(x)\) - остаточный член интерполяционного полинома или погрешность интерполяционного полинома.
\vspace{5mm}
\marginpar
{
  \footnotesize \(x_0, x_1, \dots, x_m\) - узлы; $k$ - нек. пост.
}
\begin{equation}
  \varphi(z) = f(z) - Q_m(z) - k\omega(z)
\end{equation}

Пусть (.)$x$ - точка в которой оценивается погрешность, не совпадающая с узлами интерполирования.
\[\varphi(x_i) = 0, \, i = 0, 1, \dots, m\]
Выберем $k$ таким образом, чтобы $\varphi(z)$ в этой точке была равна $0$.
\begin{equation}
  \varphi(x) = 0 \Rightarrow k = \frac{f(x) - Q_m(x)}{\omega(x)}
\end{equation}
Таким образом, $\varphi(z)$ имеет по меньшей мере $m+2$ нуля: $x_0, x_1, \dots, x_m, x$. \\
\begin{importantblock}
  \textbf{Теорема Ролля.}
  Если $f(x)$ непрерывна на отрезке $[a,b]$, имеет первую производную в каждой точке внутри этого отрезка, и значения функции на концах этого промежутка равны, т.е. $f(a)=f(b)$, то внутри отрезка найдется, по крайней мере, одна такая точка $x=c$, что $f^{'}(c)=0$.
\end{importantblock}
Тогда по теореме Ролля:
\marginpar
{
  \footnotesize Произв. $\rightarrow$ мин. кол-во нулей \\

  \footnotesize Точку, где $(m+1)$-я произв. 0, об. за $\eta$.
}
\begin{align*}
  \varphi^{'}(z) & \rightarrow m + 1 \text{нуль} \\
  \varphi^{''}(z) & \rightarrow m \\
  \dots \\
  \varphi^{(m + 1)}(z) & \rightarrow 1 \Rightarrow \varphi^{(m + 1)}(\eta) = 0
\end{align*}

Продифференцируем $\varphi(z)$ $m+1$ раз и подставим $\eta$:
\[0 = \varphi^{(m+1)}(\eta) = f^{(m+1)}(\eta) - 0 - k * (m + 1)! \Rightarrow k = \frac{f^{(m+1)}(\eta)}{(m+1)!}\]
Подставляем выражение для $k$ в (3):
\[f(x) = Q_m(x) + \boxed{\frac{\omega(x)}{(m+1)!}f^{(m+1)}(\eta)}\] \\
\textit{Выделенная часть и является погрешностью интерполяционного полинома.}
Точка $\eta$ зависит от:
\begin{itemize}
  \item вида функции $f$,
  \item выбора узлов интерполирования,
  \item точки, в которой оценивается погрешность.
\end{itemize}

\end{document}

\documentclass[a4paper,11pt]{article}

\usepackage{wrapfig}
\usepackage{amsmath}
\usepackage{pgfplots}
\usepackage{enumerate}
\usepackage{cancel}
\usepackage[most]{tcolorbox} % для управления цветом
%Russian-specific packages
%--------------------------------------
\usepackage[T2A]{fontenc}
\usepackage[utf8]{inputenc}
\usepackage[russian, english]{babel}
%--------------------------------------

%Для выделения блока текста в рамку
\definecolor{block-gray}{gray}{0.95} % уровень прозрачности (1 - максимум)
\newtcolorbox{importantblock}{colback=block-gray,grow to right by=-10mm,grow to left by=-10mm,
boxrule=0pt,boxsep=0pt,breakable} % настройки области с изменённым фоном

\definecolor{lemonchiffon}{rgb}{1.0, 0.98, 0.8}
\newtcolorbox{mainblock}{colback=lemonchiffon,grow to right by=-10mm,grow to left by=-10mm,
boxrule=0pt,boxsep=0pt,breakable} % настройки области с изменённым фоном

\makeatletter
\newcommand{\settag}[1]{
  \tag*{(#1)\qquad}
  \edef\@currentlabel{\theequation#1}}
\makeatother

\title{21. Метод последовательных приближений для решения линейных систем}
\author{Андрей Бареков \and Ярослав Пылаев \and По лекциям Устинова С.М.}
\date{\today}

\begin{document}
\maketitle
\newpage

\noindent Методы последовательных приближений (\textit{итерационные} методы) дают возможность для системы $Ax=b$ строить последовательность
векторов $x_0, x_1, \dots, x_n, \dots$, пределом которой должно быть точное решение $x^*$: \[x^*=\lim_{n\rightarrow\infty x_n}.\]
\begin{equation}
  Ax = b, \hspace{3mm} det(A) \ne 0.
  \label{eq:SysLineEq}
\end{equation}
Эквивалентными преобразованиями система (\ref{eq:SysLineEq}) сводится к виду
\begin{equation}
  x = Cx+d.
  \label{eq:NewSys}
\end{equation}
Точное решение системы (\ref{eq:NewSys}) имеет вид:
\begin{equation}
  x^* = (E-C)^{-1}d,
  \label{eq:AccurateSole}
\end{equation}
\begin{center}
  $x^* = x_n+\varepsilon_n.$
\end{center}
Вместо системы (\ref{eq:NewSys}) будем решать систему разностных уравнений
\begin{equation}
  x_{n+1} = Cx_n+d
  \label{eq:DiffEq}
\end{equation}
пошаговым методом, при $x_n \rightarrow x^* (n \rightarrow \infty)$.\\

\noindent Система (\ref{eq:DiffEq}) имеет вид системы линейных разностных уравнений (вопрос "18. Решение систем..."), поэтому ее решение записывается в виде
\begin{equation}
  x_n = C^nx_0 + (E-C^n)(E-C)^{-1}d.
  \label{eq:DiffSysSolve}
\end{equation}
Из (\ref{eq:AccurateSole}) вычитаем (\ref{eq:DiffSysSolve})
\begin{align*}
  x^*-x_n = \varepsilon_n &= -C^nx_0 + C^n\underbrace{(E-C)^{-1}d}_{x^*} \\
  &= C^n(x^*-x_0) = C^n\varepsilon_0.
\end{align*}
Для обеспечения условия сходиомости для точного решения ($\varepsilon \rightarrow 0, \hspace{2mm}\\ n \rightarrow \infty$) необходимо, чтобы элементы матрицы $C^n$
при $n \rightarrow \infty$ стремились бы к $0$. Для этого необходимо и достаточно, чтобы все собственные значения матрицы $C^n$
были по модулю меньше единицы. \\
Если все собственные значения матрицы различны, то по \textit{теореме 7}:
\begin{equation*}
  C^n = \sum_{k=1}^N T_k\lambda_k^n,
\end{equation*}
Условие сходиомости:
\begin{equation}
  |\lambda_k|<1, \hspace{3mm} k=1,2,\dots,N.
  \label{eq:NesCond}
\end{equation}
Из \textit{следствия 2} к \textit{теореме 5} \[|\lambda_k|\le\parallel C\parallel,\]
поэтому на практике вместо необходимого и достаточного условия (\ref{eq:NesCond}) используют достаточное условие
\begin{equation}
  \parallel C\parallel < 1.
  \label{eq:PlentyCond}
\end{equation}
Процесс сходится тем быстрее, чем меньше $\lambda_k$.
Выбор $x_0$ не влияет на факт сходимости, но его удачный выбор сокращает количество итераций.
\begin{importantblock}
  Искусство пользователя заключается в том, чтобы перейти от системы (\ref{eq:SysLineEq}) к системе (\ref{eq:NewSys}) так,
  чтобы выполнялись условия (\ref{eq:NesCond}) и (\ref{eq:PlentyCond}).
\end{importantblock}

\noindent Универсального алгоритма малой трудоемкости нет, поэтому используют различные приемы для конкретного вида матриц.\\

\noindent \textit{Пример.} \\
\begin{equation*}
  Ax=b, \settag{1}
\end{equation*}
Пусть диагональные элементы матрицы $A$ по модулю значительно превышают остальные элементы соответствующих строк. \\
Разделим каждое уравнение на диагональные элементы
\begin{gather*}
  A^*x = b^*, \settag{1*} \\
  x = \underbrace{(E-A^*)}_{C}x + b^*.
\end{gather*}
$\parallel C\parallel \ll 1$, потому что на главной диагонали матрицы $A^*$ стоят единицы, а у матрицы $(E-A^*)$ расположены нули.
Вне главной диагонали у обеих матриц находятся малые по модулю элементы, что вместе и обеспечивает условия (\ref{eq:NesCond}) и (\ref{eq:PlentyCond}).

\end{document}

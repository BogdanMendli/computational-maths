\documentclass[a4paper,11pt]{article}

\usepackage{wrapfig}
\usepackage{amsmath}
\usepackage{amsfonts}
\usepackage{pgfplots}
\usepackage{xcolor}
\usepackage[most]{tcolorbox}
%Russian-specific packages
%--------------------------------------
\usepackage[T2A]{fontenc}
\usepackage[utf8]{inputenc}
\usepackage[russian, english]{babel}
%--------------------------------------

\usepgfplotslibrary{fillbetween}
\usetikzlibrary{backgrounds}
\tikzset{
    reverseclip/.style={insert path={(current page.north east) --
        (current page.south east) --
        (current page.south west) --
        (current page.north west) --
        (current page.north east)}
    }
}

\definecolor{lemonchiffon}{rgb}{1.0, 0.98, 0.8}
\newtcolorbox{mainblock}{colback=lemonchiffon,grow to right by=-10mm,grow to left by=-10mm,
boxrule=0pt,boxsep=0pt,breakable} % настройки области с изменённым фоном
\definecolor{block-gray}{gray}{0.95} % уровень прозрачности (1 - максимум)
\newtcolorbox{importantblock}{colback=block-gray,grow to right by=-10mm,grow to left by=-10mm,
boxrule=0pt,boxsep=0pt,breakable} % настройки области с изменённым фоном

\makeatletter
\newcommand{\settag}[1]{
  \tag*{(#1)\qquad}
  \edef\@currentlabel{\theequation#1}}
\makeatother

\title{27. Глобальная погрешность. Устойчивость метода. Ограничение на шаг. Явление жёсткости и методы решения жёстких систем}
\author{Андрей Бареков \and Ярослав Пылаев \and По лекциям Устинова С.М.}
\date{\today}

\begin{document}
\maketitle
\newpage

Для обеспечения малой величины глобальной погрешности необходимо обеспечить не только малую локальную погрешность, но и устойчивость метода. \\

Устойчивость методов будем анализировать на примере тестовой системы - линейной системы дифференциальных уравнений с постоянной матрицей. Нелинейная система
  в малой окрестности какой-то точки может быть хорошо аппроксимирована линейной системой, и если метод плохой для линейной системы, то и в общем случае он плохой. \\
\marginpar {
  \footnotesize \[x^{'} = Ax\] \[\Re(\lambda_k) < 0\] \[x_{n+1} = Bx_n\] \[|\lambda_k(B)| < 1\]
}
\begin{gather}
  \frac{dx}{dt} = f(t, x) \\
  \frac{dx}{dt} = Ax
  \label{eq:DESys}
\end{gather}
Пусть $\lambda_k$ матрицы $A$ лежат в левой полуплоскости, тогда решение системы (\ref{eq:DESys}) будет асимптотически устойчивым. \\
Для качественного соответствия точного и приближённого решения необходимо, чтобы решение разностного уравнения численного метода также было асимптотически устойчиво. \\
Обратимся к явному методу ломаных Эйлера:
\begin{equation}
  x_{n+1} = x_n + hf(t_n, x_n) = x_n + hAx_n = (E + hA)x_n
  \label{eq:FEM}
\end{equation}
Чтобы $x_n$ было асимптотически устойчивым, необходимо, чтобы $|\lambda_k(E + hA)| < 1$. \\
Если у матрицы $A$ - собственные значения $\lambda_k$, то у матрицы $E + hA$ собственные значения $1 + h\lambda_k$. \\
Условие принимает вид $|1 + h\lambda_k| < 1$. \\
\textbf{\textit{a).}} $\lambda_k \in \mathbb{R},\, \lambda_k < 0$.
\[-1 < 1 + h\lambda_k < 1\]
Правое неравенство соблюдается всегда, левое:
\begin{equation}
  -h\lambda_k < 2 \Leftrightarrow h < \frac{2}{|\lambda_k|_{max}}
  \label{eq:SSR}
\end{equation}
Т.к. $|\lambda_k| \le \|A\|$, то на практике часто используют достаточное условие устойчивости \[h \le \frac{2}{\|a\|} \settag{4*}\]
\textbf{\textit{б).}} $\lambda_k = \alpha + i\omega,\, \alpha < 0$.
\begin{align}
    |1 + h\lambda_k|              &< 1 \notag \\
    |1 + h\alpha + ih\omega|      &< 1 \notag \\
    (1 + h\alpha)^2 + h^2\omega^2 &< 1
\end{align}
\begin{center}
  \begin{tikzpicture}
    \begin{axis} [
      xlabel = {$h\alpha$}, ylabel = {$h\omega$},
      xmin = -3, ymin = -2, xmax = 2, ymax = 2,
      xtick = {}, ytick = {},
      axis equal,
      axis x line = middle, axis y line = middle,
      no markers
    ]
      \addplot+[fill=orange, fill opacity=0.2] [
        domain = -180:180,
        samples = 100,
        color = orange,
        style = dashed
      ] ({cos(x)-1},{sin(x)}); 
    \end{axis}
  \end{tikzpicture}
\end{center}
Множество значений $h\lambda$, удовлетворяющее условию устойчивости, называется областью устойчивости данного метода. \\
Является ли ограничение на шаг (\ref{eq:SSR}) обременительным на практике? \\
\marginpar {
  \footnotesize \[e^{-3} < 0.05\] \[e^{-5} < 0.01\]
}
\textbf{\underline{Пример 1.}} $\lambda_1 = -1, \lambda_2 = -2$
\[x(t) = C_1e^{-t} + C_2e^{-2t}\]
Будем строить график на промежутке $t \in [0; T],\, T = 3$. Формула (\ref{eq:SSR}) даёт $h < 1$.
\marginpar{\footnotesize $h = 0.1 - 0.5$ \\ П. $h = 0.1$} \\
\textit{Вывод:} условие (\ref{eq:SSR}) проблем не создаёт. \\
\textbf{\underline{Пример 2.}} $\lambda_1 = -1, \lambda_2 = -20000$
\[x(t) = C_1e^{-t} + C_2e^{-2\cdot 10^4t}\]
$t$ определяется самой "медленной" экспонентой. \\
Будем строить график на промежутке $t \in [0; T],\, T = 3$. Формула (\ref{eq:SSR}) даёт $h < 10^{-4}$. \\

Описанная ситуация будет проявляться тем острее, чем больше разброс между собственными значениями матрицы. \\
На практике $T \sim \frac{1}{|\lambda_k|_{min}},\, h < \frac{1}{|\lambda_k|_{max}}$. \\
Тогда число шагов $= \frac{T}{h} \sim \frac{|\lambda_k|_{max}}{|\lambda_k|_{min}} \gg 1$ пропорционально числу обусловленности матрицы. Чем хуже матрица
  обусловлена, тем больше шагов. \\
Такие системы дифференциальных уравнений с плохо обусловленной матрицей получили название \textbf{жёсткие системы дифференциальных уравнений}. \\
Для их решений характерны два участка:
\begin{enumerate}
  \item Участок малой продолжительности ($\tau_{\text{пс}}$) называется пограничным слоем. На нём решение меняется очень быстро и обладает большими производными.
  \item На остальной части промежутка ($T$) решение меняется относительно медленно: $\tau_{\text{пс}} \ll T$
\end{enumerate}
Рассмотрим метод Эйлера-Коши:
\[x_{n+1}^* = x_n + hf(t_n, x_n)\]
\begin{align*}
  x_{n+1} &= x_n + \frac{h}{2}(f(t_n, x_n) + f(t_{n+1}, x_{n+1}^*)) \\
          &= x_n + \frac{h}{2}(Ax_n + A(x_n + hAx_n)) \\
          &= x_n(E + hA + \frac{h^2A^2}{2})
\end{align*}
\begin{align*}
  \Big|\lambda_k(E + hA + \frac{h^2A^2}{2})\Big| &< 1 \\
  \Big|1 + h\lambda_k + \frac{h^2\lambda_k^2}{2}\Big| &< 1
\end{align*}
Ограничимся случаем, когда $\lambda_k$ вещественны.
\[-1 < 1 + h\lambda_k + \frac{h^2\lambda_k^2}{2} < 1\]
Левое неравенство выполняется всегда,\footnote{$h\lambda_k$ всегда отрицательно, после преобразований можно получить выражение $-4-2h\lambda_k < (h\lambda_k)^2$, 
                                              что при любых значениях $h$ верно.} правое:
\begin{gather*}
  h\lambda_k + \frac{h^2\lambda_k^2}{2} < 0 \\
  1 + \frac{h\lambda_k}{2} > 0 \Leftrightarrow 2 > -h\lambda_k
\end{gather*}
Выходит то же самое, что и в (\ref{eq:SSR}), хоть и будут различия, при переходе в $\mathbb{C}$. \\
Аналогичные ограничения имеют все явные методы Рунге-Кутты и Адамса:
\begin{align*}
  RK2    &\rightarrow h|\lambda_k| < 2 \\
  RK4    &\rightarrow h|\lambda_k| < 2.785\dots \\
  Adams2 &\rightarrow h|\lambda_k| < 1 \\
  Adams4 &\rightarrow h|\lambda_k| < 0.3
\end{align*}
Для решения жёстких систем хотелось бы применять методы, область устойчивости которых включала бы в себя всю или почти всю левую полуплоскость. \\
Рассмотрим неявный метод ломаных Эйлера:
\[x_{n+1} = x_n + hf(t_{n+1}, x_{n+1}) = x_n + hAx_{n+1}\]
\begin{gather*}
  (E - hA)x_{n+1} = x_n \\
  x_{n+1} = (E - hA)^{-1}x_n
\end{gather*}
Пусть $\lambda_k = \alpha + i\omega$.
\begin{align*}
  |\lambda_k(E - hA)^{-1}| &< 1 \\
  \frac{1}{|1 - h\lambda_k|} &< 1 \\
  |1 - h\lambda_k| &> 1 \\
  |1 - h\alpha - ih\omega| &> 1 \\
  (1 - h\alpha)^2 + h^2\omega^2 &> 1
\end{align*}
\begin{center}
  \begin{tikzpicture}
    \begin{axis} [
      xlabel = {$h\alpha$}, ylabel = {$h\omega$},
      xmin = -2, ymin = -2, xmax = 3, ymax = 2,
      xtick = {}, ytick = {},
      axis equal, axis on top = true,
      axis x line = middle, axis y line = middle,
      no markers
    ]
      \fill[orange, opacity=0.2] (axis cs: -2, -2) rectangle (axis cs: 3, 2);
      \addplot [
        domain = -180:180,
        samples = 100,
        color = orange,
        style = dashed
      ] ({cos(x)+1},{sin(x)});
      \addplot+[fill=white] [
        domain = -180:180,
        samples = 100,
        color = white,
        style = dashed, draw = none
      ] ({cos(x)+1},{sin(x)});
    \end{axis}
  \end{tikzpicture}
\end{center}
Для $\lambda_k < 0$ ограничение по устойчивости отсутствует, и он может выбираться только из соображений точности. Больший объём вычислений на шаге по сравнению
  с явным методом для жёстких систем с лихвой окупается выигрышем в величине шага. \\
Рассмотрим неявный метод трапеций:
\[x_{n+1} = x_n + \frac{h}{2}(f(t_n, x_n) + f(t_{n+1}, x_{n+1})) = x_n + \frac{h}{2}(Ax_n + Ax_{n+1})\]
\begin{gather*}
  \Big(E - \frac{hA}{2}\Big)x_{n+1} = \Big(E + \frac{hA}{2}\Big)x_n \\
  x_{n+1} = \Big(E - \frac{hA}{2}\Big)^{-1}\Big(E + \frac{hA}{2}\Big)x_n
\end{gather*}
Пусть $\lambda_k = \alpha + i\omega$.
\begin{align*}
  \bigg|\lambda_k\bigg(\Big(E - \frac{hA}{2}\Big)^{-1}\Big(E + \frac{hA}{2}\Big)\bigg)\bigg| &< 1 \\
  \frac{\Big|1 + \frac{h\lambda_k}{2}\Big|}{\Big|1 - \frac{h\lambda_k}{2}\Big|} &< 1 \\
  \Big|1 + \frac{h\lambda_k}{2}\Big| &< \Big|1 - \frac{h\lambda_k}{2}\Big| \\
  \Big|1 + \frac{h\alpha}{2} + \frac{ih\omega}{2}\Big| &< \Big|1 - \frac{h\alpha}{2} - \frac{ih\omega}{2}\Big| \\
  (1 + \frac{h\alpha}{2})^2 + \underline{\frac{h^2\omega^2}{4}} &< (1 - \frac{h\alpha}{2})^2 + \underline{\frac{h^2\omega^2}{4}} \\
  \underline{1} + h\alpha + \underline{\frac{h^2\alpha^2}{4}} &< \underline{1} - h\alpha + \underline{\frac{h^2\alpha^2}{4}} \\
  2h\alpha &< 0 \\
  h\alpha &< 0
\end{align*}
Область устойчивости метода трапеций в точности равна левой полуплоскости. \\
Методы, пригодные для решения жёстких систем, как правило, являются неявными.

\end{document}

\documentclass[a4paper,11pt]{article}

\usepackage{wrapfig}
\usepackage{amsmath}
\usepackage{pgfplots}
\usepackage{xcolor}
\usepackage[most]{tcolorbox}
%Russian-specific packages
%--------------------------------------
\usepackage[T2A]{fontenc}
\usepackage[utf8]{inputenc}
\usepackage[russian, english]{babel}
%--------------------------------------

\definecolor{lemonchiffon}{rgb}{1.0, 0.98, 0.8}
\newtcolorbox{mainblock}{colback=lemonchiffon,grow to right by=-10mm,grow to left by=-10mm,
boxrule=0pt,boxsep=0pt,breakable} % настройки области с изменённым фоном
\definecolor{block-gray}{gray}{0.95} % уровень прозрачности (1 - максимум)
\newtcolorbox{importantblock}{colback=block-gray,grow to right by=-10mm,grow to left by=-10mm,
boxrule=0pt,boxsep=0pt,breakable} % настройки области с изменённым фоном

\makeatletter
\newcommand{\settag}[1]{
  \tag*{(#1)\qquad}
  \edef\@currentlabel{\theequation#1}}
\makeatother

\title{18. Решение систем линейных дифференциальных и разностных уравнений с постоянной матрицей.}
\author{Андрей Бареков \and Ярослав Пылаев \and По лекциям Устинова С.М.}
\date{\today}

\begin{document}
\maketitle
\newpage

\section{Системы линейных дифференциальных уравнений}
\begin{importantblock}
  \begin{equation}
    \frac{dx}{dt} = Ax + f(t),\, x(0) = x_0)
    \label{eq:LDLESys}
  \end{equation}
  $A$ - заданная постоянная квадратная матрица, \\
  $f(t)$ - заданная вектор-функция, \\
  $f(x)$ - вектор-решение.
\end{importantblock}
Начинаем решать с однородной системы:
\begin{equation}
  \frac{dx}{dt} = Ax
  \label{eq:LDLEHomo}
\end{equation}
\marginpar {
  \begin{gather*}
    \footnotesize x^{'} = \alpha x \\
    \Rightarrow x(t) = e^{\alpha t} \times C
  \end{gather*}
}
\begin{equation}
  \underbrace{x(t)}_{\text{вектор}} = \underbrace{e^{At}}_{\text{матрица}} \times \underbrace{C}_{\text{вектор}}
  \label{eq:LDLEHomoSol}
\end{equation}
Подставляем уравнение (\ref{eq:LDLEHomoSol}) в уравнение (\ref{eq:LDLEHomo}):
\[Ae^{At}\times C = Ae^{At}\times C\]
Решение неоднородной системы (\ref{eq:LDLESys}) в соответствии с методом Лагранжа будем искать в виде:
\begin{equation}
  x(t) = e^{At} \times C(t)
  \label{eq:LDLESol}
\end{equation}
Подставляем уравнение (\ref{eq:LDLESol}) в уравнение (\ref{eq:LDLESys}):
\[\underline{Ae^{At}\times C(t)} + e^{At}\frac{dC(t)}{dt} = \underline{Ae^{At}\times C(t)} + f(t)\]
Умножим на обратную матрицу слева:
\[\frac{dC(t)}{dt} = e^{-At}f(t)\]
\begin{equation}
  C(t) = C(0) + \int_0^t e^{-A\tau}f(\tau)\, d\tau
  \label{eq:LDLECt}
\end{equation}
Подставляем уравнение (\ref{eq:LDLECt}) в уравнение (\ref{eq:LDLESol}):
\[x(t) = e^{At}C(0) + e^{At}\int_0^t e^{-A\tau}f(\tau)\,d\tau = e^{At}C(0)+\int_0^t e^{A(t-\tau)}f(\tau)\,d\tau\]
Полагаем, что $t = 0$:
\[x(0) = E\times C(0) + \underbrace{\int_0^0}_{0} \Rightarrow C(0) = x_0\]
\begin{equation}
  x(t) = e^{At} \times x_0 + \int_0^t e^{A(t-\tau)}f(\tau)\,d\tau
  \label{eq:LDLEExact}
\end{equation}
Формула (\ref{eq:LDLEExact}) - точное решение системы (\ref{eq:LDLESys}).

\section{Системы линейных разностных уравнений}
\setcounter{equation}{0}
\begin{importantblock}
  \begin{equation}
    x_{n+1} = Bx_n + \varphi(n),\, x_n |_{n=0} = x_0
    \label{eq:LDESys}
  \end{equation}
  $B$ - заданная постоянная квадратная матрица, \\
  $\varphi(n)$ - заданная вектор-функция, \\
  $x_n$ - вектор-решение.
\end{importantblock}
\begin{align*}
  x_1 &= Bx_0 + \varphi_0 \\
  x_2 &= Bx_1 + \varphi_1 = B^2x_0 + B\varphi_0 + \varphi_1 \\
  x_3 &= Bx_2 + \varphi_2\dots \\
  & \text{по индукции:} \\
  x_n &= B^nx_0 + \sum_{k=0}^{n-1} B^k\varphi_{n-1-k} \settag{2}
\end{align*}
Рассмотрим важный частный случай, когда \(\varphi(n) = \alpha - const\):
\[x_n = B^nx_0 + (\sum_{k=0}^{n-1} B^k) \times \alpha\]
\marginpar {
  \footnotesize (*) на осн. теор. 4 о МФ возм. оба варианта.
}
\begin{equation*}
  \sum_{k=0}^{n-1} B^k \, \footnote{$\sum_{k=0}^{n-1} B^k = \frac{1-B^n}{1-B}$}
  \begin{array}{c}
    \nearrow (E-B^n)(E-B)^{-1}\\
    \searrow (E-B)^{-1}(E-B^n)
  \end{array}^{(*)}
\end{equation*}
\begin{equation*}
  x_n = B^nx_0 + (\sum_{k=0}^{n-1} B^k) \times \alpha = B^nx_0 + (E-B^n)(E-B)^{-1}\alpha \settag{2*}
\end{equation*}
\end{document}

\documentclass[a4paper,11pt]{article}

\usepackage{wrapfig}
\usepackage{amsmath}
\usepackage{pgfplots}
\usepackage{xcolor}
\usepackage[most]{tcolorbox}
%Russian-specific packages
%--------------------------------------
\usepackage[T2A]{fontenc}
\usepackage[utf8]{inputenc}
\usepackage[russian, english]{babel}
%--------------------------------------

\definecolor{lemonchiffon}{rgb}{1.0, 0.98, 0.8}
\newtcolorbox{mainblock}{colback=lemonchiffon,grow to right by=-10mm,grow to left by=-10mm,
boxrule=0pt,boxsep=0pt,breakable} % настройки области с изменённым фоном
\definecolor{block-gray}{gray}{0.95} % уровень прозрачности (1 - максимум)
\newtcolorbox{importantblock}{colback=block-gray,grow to right by=-10mm,grow to left by=-10mm,
boxrule=0pt,boxsep=0pt,breakable} % настройки области с изменённым фоном

\makeatletter
\newcommand{\settag}[1]{
  \tag*{(#1)\qquad}
  \edef\@currentlabel{\theequation#1}}
\makeatother

\title{09. Квадратурные формулы левых, правых, и средних прямоугольников, трапеций, Симпсона. Малые и составные формулы, их остаточные члены.}
\author{Андрей Бареков \and Ярослав Пылаев \and По лекциям Устинова С.М.}
\date{\today}

\begin{document}
\maketitle
\newpage

\section{Квадратурные формулы}
Так называются формулы для вичисления определённых интегралов.
\begin{gather*}
  I = \int_a^b f(x)\ dx \\
  Q_0(x) = f(x_0) \\
  I \approx \int_a^b f(x_0)\ dx = (b - a)f(x_0)
\end{gather*}
  \begin{gather*}
    f(x) = Q_m(x) + R_m(x) \\
    \int_a^b f(x)\ dx = \int_a^b Q_m(x)\ dx \\
    \varepsilon = \int_a^b R_m(x)\ dx
  \end{gather*}
\begin{tcolorbox}
  \begin{equation}
    x_0 = a \Rightarrow I \approx (b - a)f(a)
    \label{eq:QRLR}
  \end{equation}
  \begin{center}
    \begin{tikzpicture}
      \begin{axis} [
        xlabel = {}, ylabel = {},
        xmin = 0, ymin = 0,
        xtick = {}, ytick = {},
        no markers
      ]

        \addplot+[smooth] [
        color = blue,
        thick,
        domain = 0:4,
        samples = 300
      ] coordinates { (0,3)(0.25,3.4)(0.5,3.7)(0.75,3.9)(1,4.05)(1.25,4.2)(1.5,4.35)(1.75,4.5)(2,4.6)(2.25,4.7)(2.5,4.8)(2.75,4.9)(3,5) };

        \addplot+[smooth, fill=orange, fill opacity=0.1] [
        color = orange,
        ultra thin,
        domain = 0:3,
        samples = 10
      ] { 3 } \closedcycle;

      \end{axis}
    \end{tikzpicture} \\
    Квадратурная формула левых прямоугольников.
  \end{center}
\end{tcolorbox}

\begin{tcolorbox}
  \begin{equation}
    x_0 = b \Rightarrow I \approx (b - a)f(b)
    \label{eq:QRRR}
  \end{equation}
  \begin{center}
    \begin{tikzpicture}
      \begin{axis} [
        xlabel = {}, ylabel = {},
        xmin = 0, ymin = 0,
        xtick = {}, ytick = {},
        no markers
      ]

        \addplot+[smooth] [
        color = blue,
        thick,
        domain = 0:4,
        samples = 300
      ] coordinates { (0,3)(0.25,3.4)(0.5,3.7)(0.75,3.9)(1,4.05)(1.25,4.2)(1.5,4.35)(1.75,4.5)(2,4.6)(2.25,4.7)(2.5,4.8)(2.75,4.9)(3,5) };

        \addplot+[smooth, fill=orange, fill opacity=0.1] [
        color = orange,
        ultra thin,
        domain = 0:3,
        samples = 10
      ] { 5 } \closedcycle;

      \end{axis}
    \end{tikzpicture} \\
    Квадратурная формула правых прямоугольников.
  \end{center}
\end{tcolorbox}

\begin{tcolorbox}
  \begin{equation}
    x_0 = \frac{a+b}{2} \Rightarrow I \approx (b - a)f(\frac{a+b}{2})
    \label{eq:QRMR}
  \end{equation}
  \begin{center}
    \begin{tikzpicture}
      \begin{axis} [
        xlabel = {}, ylabel = {},
        xmin = 0, ymin = 0,
        xtick = {}, ytick = {},
        no markers
      ]

        \addplot+[smooth] [
        color = blue,
        thick,
        domain = 0:4,
        samples = 300
      ] coordinates { (0,3)(0.25,3.4)(0.5,3.7)(0.75,3.9)(1,4.05)(1.25,4.2)(1.5,4.35)(1.75,4.5)(2,4.6)(2.25,4.7)(2.5,4.8)(2.75,4.9)(3,5) };

        \addplot+[smooth, fill=orange, fill opacity=0.1] [
        color = orange,
        ultra thin,
        domain = 0:3,
        samples = 10
      ] { 4.35 } \closedcycle;

      \end{axis}
    \end{tikzpicture} \\
    Квадратурная формула средних прямоугольников.
  \end{center}
\end{tcolorbox}

\begin{tcolorbox}
  \[Q_1(x) = \frac{x-b}{a-b}f(a)+\frac{x-a}{b-a}f(b)\]
  \begin{equation}
    I \approx \int_a^b Q_1(x)\ dx = \frac{b-a}{2}(f(a) + f(b))
    \label{eq:QRT}
  \end{equation}
  \begin{center}
    \begin{tikzpicture}
      \begin{axis} [
        xlabel = {}, ylabel = {},
        xmin = 0, ymin = 0,
        xtick = {}, ytick = {},
        no markers
      ]

        \addplot+[smooth] [
        color = blue,
        thick,
        domain = 0:4,
        samples = 300
      ] coordinates { (0,3)(0.25,3.4)(0.5,3.7)(0.75,3.9)(1,4.05)(1.25,4.2)(1.5,4.35)(1.75,4.5)(2,4.6)(2.25,4.7)(2.5,4.8)(2.75,4.9)(3,5) };

        \addplot+[smooth, fill=orange, fill opacity=0.1] [
        color = orange,
        ultra thin,
        domain = 0:3,
        samples = 10
      ] coordinates { (0,3)(3,5) } \closedcycle;

      \end{axis}
    \end{tikzpicture} \\
    Квадратурная формула трапеций.
  \end{center}
\end{tcolorbox}

\begin{tcolorbox}
  \[Q_2(x); x_0 = a, x_1 = \frac{a+b}{2}, x_2 = b\]
  \begin{equation}
    I \approx \int_a^b Q_2(x)\ dx = \frac{b-a}{6}(f(a) + 4f(\frac{a+b}{2}) + f(b))
    \label{eq:QRS}
  \end{equation}
  \begin{center}
    \begin{tikzpicture}
      \begin{axis} [
        xlabel = {}, ylabel = {},
        xmin = 0, ymin = 0,
        xtick = {}, ytick = {},
        no markers
      ]

        \addplot+[smooth] [
        color = blue,
        thick,
        domain = 0:4,
        samples = 300
      ] coordinates { (0,3)(0.25,3.4)(0.5,3.7)(0.75,3.9)(1,4.05)(1.25,4.2)(1.5,4.35)(1.75,4.5)(2,4.6)(2.25,4.7)(2.5,4.8)(2.75,4.9)(3,5) };

        \addplot+[smooth, fill=orange, fill opacity=0.1] [
        color = orange,
        ultra thin,
        domain = 0:3,
        samples = 10
      ] { (x-1.5)*(x-3)*2/3 - 4.35*x*(x-3)/2.25 + 5*x*(x-1.5)/4.5 } \closedcycle;

      \end{axis}
    \end{tikzpicture} \\
    Метод параболы или квадратурная формула Симпсона.
  \end{center}
\end{tcolorbox}
Для повышения точности исходный промежуток разбивается на участки, на каждом из них применяется какая-то квадратурная формула, а результаты складываются.
Эти квадратурные формулы получили название

\section{Составные квадратурные формулы}
Разобьём исходный промежуток на $m$ участков одинаковой длины: \\
\([x_0, x_1], [x_1, x_2], \dots, [x_k, x_{k+1}], \dots, [x_{N-1}, x_N]*\)
\marginpar
{
  \footnotesize (*)
  \begin{gather*}
    \footnotesize x_k = x_0 + k\frac{b-a}{N} \\
    x_0 = a, x_N = b \\
    x_{k+1} - x_k = \frac{b-a}{N}
  \end{gather*}
}
На каждом участке вычисляем интеграл:
\[I_k = \int_{x_k}^{x_{k+1}} f(x)\ dx \approx \frac{b-a}{N}f(x_k)\]
\begin{importantblock}
  \[I = \int_a^b f(x)\ dx\]
  \begin{align*}
    \stepcounter{equation}
    I & \approx \frac{b-a}{N} \sum_{k=0}^{N-1} f(x_k) \settag{1*} \\
    I & \approx \frac{b-a}{N} \sum_{k=1}^{N} f(x_k) \settag{2*} \\
    I & \approx \frac{b-a}{N} \sum_{k=0}^{N-1} f(x_k + \frac{b-a}{2N}) \settag{3*}
  \end{align*}
\end{importantblock}
\[I_k = \int_{x_k}^{x_{k+1}} f(x)\ dx \approx \frac{b-a}{2N}(f(x_k) + f(x_{k+1}))\]
\begin{importantblock}
  \begin{align*}
    \stepcounter{equation}
    I & \approx \frac{b-a}{2N} \sum_{k=1}^{N-1} f(x_k) \settag{4*}
  \end{align*}
\end{importantblock}
В формуле Симпсона выберем $N$ чётных. Количество промежутков будет $N/2$, каждый промежуток будет вдвое большей длины.
\[I_k = \int_{x_k}^{x_{k+2}} f(x)\ dx \approx \frac{b-a}{3N}(f(x_k) + 4f(x_{k+1}) + f(x_{k+2}))\]
\begin{importantblock}
  \begin{align*}
    \stepcounter{equation}
    I & \approx \frac{b-a}{3N}(f(a) + 4(f_1 + f_3 + \dots + f_{N-1}) \\
      & + 2(f_2 + f_4 + \dots + f_{N-2}) + f(b)) \settag{5*}
  \end{align*}
\end{importantblock}

\section{Погрешности квадратурных формул}
Теорема о 'средней' точке:
\[\int_a^b f(x)g(x)\ dx = g(c) \int_a^b f(x)\ dx, c \in [a, b]\]
Чтобы теорема работала, функция, стоящая под знаком интеграла, должна быть знакопостоянна.
Теорема о 'средней' точке-2:
\[\frac{1}{N}\sum_{k=1}{N}f(x_k) = f(\eta)\]

  \subsection{Погрешности несоставных формул}
  \[\varepsilon = \int_a^b R_m(x)\ dx\]
  Погрешность равна интегралу от остаточного члена интерполяционного полинома.
  \[R_0(x) = \frac{x-a}{1!}f^{'}(\eta)\]
  \begin{align*}
    \varepsilon_{\text{л.пр.}}  &= \int_a^b \frac{x-a}{1!} f^{'}(\eta)\ dx = \int_a^b (x-a)\ dx\ f^{'}(\eta^*) = \frac{(b-a)^2}{2}f^{'}(\eta^*) \\
    \varepsilon_{\text{пр.пр.}} &= \int_a^b \frac{x-b}{1!} f^{'}(\eta)\ dx = - \frac{(b-a)^2}{2}f^{'}(\eta^*) \\
    \varepsilon_{\text{трап.}}  &= \int_a^b \frac{(x-a)(x-b)}{2!} f^{''}(\eta)\ dx = - \frac{(b-a)^2}{12}f^{''}(\eta^*)
  \end{align*}
  \[\varepsilon_{\text{ср.пр.}} = \int_a^b \frac{x-\frac{a+b}{2}}{1!} f^{'}(\eta)\ dx\]
  Теорема о 'среднем' здесь не работает, используется другой способ.
  Рассмотрим разложение функции в степенной ряд Тейлора до $^2$  в точке $\frac{a+b}{2}$:
  \[f(x) = f(\frac{a+b}{2}) + \frac{x-\frac{a+b}{2}}{1!} f^{'}(\frac{a+b}{2}) + \frac{(x-\frac{a+b}{2})}{2!} f^{''}(\eta)\]
  Проинтегрируем слева и справа на $[a; b]$:
  \[\int_a^b f(x)\ dx = (b-a)f(\frac{a+b}{2}) + \textbf{0} + \underbrace{\int_a^b \frac{(x-\frac{a+b}{2})^2}{2!} f^{''}(\eta)\ dx}_{\varepsilon_{\text{ср.пр.}}}\]
  \[\varepsilon_{\text{ср.пр.}} = \int_a^b \frac{(x-\frac{a+b}{2})^2}{2!} f^{''}(\eta)\ dx = \frac{(b-a)^3}{24}f^{''}(\eta^*)\]
  \[\varepsilon_{\text{симп.}} = \int_a^b \frac{(x-a)(x-\frac{a+b}{2})(x-b)}{3!} f^{'''}(\eta)\ dx\]
  Теорема о 'среднем' не работает, погрешность выводится по другому.
  \[\varepsilon_{\text{симп.}} = - \frac{(b-a)^5}{2880} f^{''''}(\eta^*)\]
  \begin{importantblock}
    \begin{align*}
      \varepsilon_{\text{л.пр.}}  &= \frac{(b-a)^2}{2}f^{'}(\eta^*) \\
      \varepsilon_{\text{пр.пр.}} &= - \frac{(b-a)^2}{2}f^{'}(\eta^*) \\
      \varepsilon_{\text{трап.}}  &= - \frac{(b-a)^2}{12}f^{''}(\eta^*) \\
      \varepsilon_{\text{ср.пр.}} &= \frac{(b-a)^3}{24}f^{''}(\eta^*) \\
      \varepsilon_{\text{симп.}}  &= - \frac{(b-a)^5}{2880} f^{''''}(\eta^*)
    \end{align*}
  \end{importantblock}
  
  \subsection{Погрешности составных формул}
  Погрешность составной формулы равна сумме погрешностей, допущенных на отдельных участках. \\
  \boxed {
    \varepsilon_{\text{л.пр.}}^{\text{сост.}}:
  }
  \[[x_k, x_{k+1}] \rightarrow \frac{(x_{k+1} - x_k)^2}{2} f{'}(\eta_k^*) = \frac{(b-a)^2}{2N^2} f^{'}(\eta_k^*)\]
  \begin{align*}
    \varepsilon_{\text{л.пр.}}^{\text{сост.}} &= \sum_{k=0}^{N-1} \frac{(b-a)^2}{2N^2}f^{'}(\eta_k^*) \\
                                              &= \frac{(b-a)^2}{2N} \times \frac{1}{N} \sum_{k=0}^{N-1} f^{'}(\eta_k^*) \settag{1**}\\
                                              &= \frac{(b-a)^2}{2N} f^{'}(\eta^{**}) 
  \end{align*}
  \boxed {
    \text{Аналогично:}
  }
  \begin{equation*}
    \varepsilon_{\text{пр.пр.}}^{\text{сост.}} = - \frac{(b-a)^2}{2N} f^{'}(\eta^{**}) \settag{2**}
  \end{equation*}
  \boxed {
    \varepsilon_{\text{трап.}}^{\text{сост.}}:
  }
  \[[x_k, x_{k+1}] \rightarrow - \frac{(x_{k+1} - x_k)^3}{12} f{''}(\eta_k^*) = - \frac{(b-a)^3}{12N^3} f^{''}(\eta_k^*)\]
  \begin{align*}
    \varepsilon_{\text{трап.}}^{\text{сост.}} = - \frac{(b-a)^3}{12N^2} \times \frac{1}{N} \sum_{k=0}^{N-1} f^{''}(\eta_k^*) = - \frac{(b-a)^3}{12N^2} f^{''}(\eta^{**}) \settag{3**}
  \end{align*}
  \boxed {
    \text{Аналогично:}
  }
  \begin{equation*}
    \varepsilon_{\text{ср.пр.}}^{\text{сост.}} = \frac{(b-a)^3}{24N^2} f^{''}(\eta^{**}) \settag{4**}
  \end{equation*}
  \boxed {
    \varepsilon_{\text{симп.}}^{\text{сост.}}:
  }
  \[[x_k, x_{k+2}] \rightarrow - \frac{(x_{k+2} - x_k)^5}{2880} f{(4)}(\eta_k^*) = - \frac{(b-a)^5}{90N^5} f^{(4)}(\eta_k^*)\]
  \begin{align*}
    \varepsilon_{\text{симп.}}^{\text{сост.}} = - \frac{(b-a)^5}{180N^4} \times \frac{1}{N/2} \sum_{k=1}^{\frac{N}{2}} f^{(4)}(\eta_k^*) = - \frac{(b-a)^5}{180N^4} f^{(4)}(\eta^{**}) \settag{5**}
  \end{align*}
  Если продолжать увеличивать степень интерполяционного полинома, то погрешность составных формул имеет следующий вид:
  \begin{importantblock}
    \[\varepsilon = \alpha \frac{(b-a)^{S+1}}{N^S} f^{(S)}(\eta), \alpha = const\]
  \end{importantblock}

  \begin{mainblock}
    На практике погрешность оценивается одним из следующих способов:
    \begin{enumerate}
      \item По формулам выше (используется крайне редко).
      \item Наиболее популярный - вычисляют интеграл по какой-то квадратурной формуле для $N$ и $2N$, результаты сравнивают. Если погрешность велика, $N$ удваивается.
      \item В тех случаях, когда увеличение $N$ невозможно, сравнивают две разные квадратурные формулы с одним $N$.
    \end{enumerate}
  \end{mainblock}

\end{document}

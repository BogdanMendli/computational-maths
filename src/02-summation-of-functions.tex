\documentclass[a4paper,11pt]{article}

\usepackage{wrapfig}
\usepackage{amsmath}
%Russian-specific packages
%--------------------------------------
\usepackage[T2A]{fontenc}
\usepackage[utf8]{inputenc}
\usepackage[russian, english]{babel}
%--------------------------------------

\title{02. Суммирование функций. Формула Абеля суммирования по частям}
\author{Андрей Бареков \and Ярослав Пылаев \and По лекциям Устинова С.М.}
\date{\today}

\begin{document}
\maketitle
\newpage

\section{Суммирование функций}
\marginpar
{
  \begin{gather*}
    \begin{subarray}{1}
      \Delta F(k)= \varphi (k) \\
    \end{subarray}
  \end{gather*}
  \footnotesize По $\varphi (k)$ ищем $F(k)$
}
\begin{equation}
  \Delta F(k) = \varphi(k)
\end{equation}

\begin{equation}
  \begin{cases}
    F_1 - F_0 = \varphi_0 \\
    F_2 - F_1 = \varphi_1 \\
    \cdots \\
    F_{n+1} - F_n = \varphi_n
  \end{cases} \Leftrightarrow F_{n+1} - F_0 = \sum_{k=0}^{n} \varphi_k
  \label{eq:NtLbFormula}
\end{equation}

Уравнение (\ref{eq:NtLbFormula}) является дискретным аналогом формулы Ньютона-Лейбница.
\subsection{Пример}
\begin{equation*}
  \begin{split}
    \sum_{k=0}^{N} a^k & = \bigg[ \Delta F(k) = a^k; \Delta a^k = a^k(a-1), a^k = \frac{\Delta a^k}{a-1}; F(k) = \frac{a^k}{a-1} \bigg] = \\
    & = \frac{a^{N+1}-1}{a-1} - \frac{1}{a-1} = \frac{a^{N+1}-1}{a-1} = \frac{1-a^{N+1}}{1-a}
  \end{split}
\end{equation*}

  \subsection {Формула Абеля суммирования по частям}
  \[\int_{a}^{b} u\,dv = uv \bigg|_{a}^{b} - \int_{a}^{b} v\,du\]
  \begin{equation}
    \begin{split}
      \int_{a}^{b} \bigg| \frac{d}{dx}(U(x)v(x)) = u(x)v(x) + U(x)\frac{dv(x)}{dx} \\
      \int_{a}^{b} u(x)v(x)\,dx = U(x)v(x) \bigg|_{a}^{b} - \int_{a}^{b} U(x)\frac{dv}{dx} \,dx
    \end{split}
  \end{equation}
  \hrule
  \marginpar
  {
    \begin{gather*}
      \begin{subarray}{1}
        (*) \\
        \Delta (f_k g_k) = f_k \Delta g_k + g_{k+1} \Delta f_k \\
        \Delta (f_k) = f_{k+1} - f_k \\
        \Delta U(k) = \sum_{i=0}^{k+1} u(i) - \\
        - \sum_{i=0}^{k} u(i) = u(k+1)
      \end{subarray}
    \end{gather*}
  }
  \begin{align*}
    \sum_{k=m}^{n} \bigg| & \Delta(U(k) v(k)) = v(k+1) \Delta U(k) + U(k) \Delta v(k) = \\
    & = [ u(k+1) v(k+1) ]^{(*)} + U(k) \Delta v(k)
  \end{align*}
  \begin{equation}
    \sum_{k=m}^{n} u(k+1) v(k+1) = U(k) v(k) \bigg|_{m}^{n+1} - \sum_{k=m}^{n} U(k) \Delta v(k)
    \label{eq:SumByParts}
  \end{equation}

  Формула (\ref{eq:SumByParts}) - формула Абеля суммирования по частям. \\

  \subsubsection{Классический пример}
    \[\sum_{k=0}^N ka^k\] \\
    Будем использовать формулу Абеля суммирования по частям: \\
    \[\sum_{k=m}^{n} u(k+1) v(k+1) = U(k) v(k) \bigg|_{m}^{n+1} - \sum_{k=m}^{n} U(k) \Delta v(k)\]
    Но, чтобы произвести замену, нам нужно свдинуть индекс $k+1$ на единицу, то есть \\
    \begin{equation*}
      \begin{split}
      v(k) = k - 1,
      u(k) = a^{k-1}
      \end{split}
    \end{equation*}
    \marginpar
    {
      \footnotesize (*)Формула геометрической прогрессии:
      \footnotesize \[S_n = \frac{b_n*q-1}{q-1}\]
    }
    Тогда получим $U(k)$
    \marginpar
    {
      \footnotesize (**)Раcписали сумму:
      \begin{gather*}
        \footnotesize \sum_{k=0}{N} \frac{1}{a} \frac{a^{k+1}-1}{a-1} = \\
        = \sum_{k=0}^{N}a^k + \sum_{k=0}^{N}\frac{1}{a(a-1)}
      \end{gather*}
    }
    \[U_k = \sum_{i=0}^{k}a^{i-1} = \frac{1}{a} \times \bigg[ \frac{a^{k+1}-1}{a-1} \bigg]^{(*)}\]

    \begin{equation*}
      \begin{split}
        \sum_{k=0}^{N}ka^k & = \frac{1}{a} \frac{a^{k+1}-1}{a-1}(k-1) \bigg|_{0}^{N+1} - \bigg[\sum_{k=0}{N} \frac{1}{a} \frac{a^{k+1}-1}{a-1} \bigg]^{(**)} = \\
        & = \frac{1}{a} \frac{a^{N+2}-1}{a-1}N + \frac{1}{a} - \frac{1}{a-1} \sum_{k=0}^{N}a^k + \frac{N+1}{a(a-1)} = \\
        & = \frac{Na^{N+2} - Na^{N+1} - a^{N+1} + a}{(a-1)^2} = \frac{Na^{N+1}}{a-1} - \frac{a^{N+1} - a}{(a-1)^2}
      \end{split}
    \end{equation*}

\end{document}

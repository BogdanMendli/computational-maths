\documentclass[a4paper,11pt]{article}

\usepackage{wrapfig}
\usepackage{amsmath}
\usepackage{pgfplots}
\usepackage{xcolor}
\usepackage[most]{tcolorbox}
%Russian-specific packages
%--------------------------------------
\usepackage[T2A]{fontenc}
\usepackage[utf8]{inputenc}
\usepackage[russian, english]{babel}
%--------------------------------------

\definecolor{lemonchiffon}{rgb}{1.0, 0.98, 0.8}
\newtcolorbox{mainblock}{colback=lemonchiffon,grow to right by=-10mm,grow to left by=-10mm,
boxrule=0pt,boxsep=0pt,breakable} % настройки области с изменённым фоном
\definecolor{block-gray}{gray}{0.95} % уровень прозрачности (1 - максимум)
\newtcolorbox{importantblock}{colback=block-gray,grow to right by=-10mm,grow to left by=-10mm,
boxrule=0pt,boxsep=0pt,breakable} % настройки области с изменённым фоном

\makeatletter
\newcommand{\settag}[1]{
  \tag*{(#1)\qquad}
  \edef\@currentlabel{\theequation#1}}
\makeatother

\title{19. Устойчивость решений дифференциальных и разностных уравнений.}
\author{Андрей Бареков \and Ярослав Пылаев \and По лекциям Устинова С.М.}
\date{\today}

\begin{document}
\maketitle
\newpage

\section{Общий случай}
\begin{importantblock}
  Общий нелинейный случай:
  \begin{equation}
    \frac{dx}{dt} = f(t, x),\, x(t_0) = x_0
    \label{eq:LSb}
  \end{equation}
\end{importantblock}
Будем изучать характер поведения решения при возмущениях в начальных условиях. При не очень сильных ограничениях на функцию $f$, на конечном промежутке
  $t \in [t_0, T]$ имеется непрерывная зависимость решения от начальных условий. Поэтому наибольший интерес вызывает поведение решения при $t \rightarrow \infty$.

\begin{importantblock}
  Решение $x(t)$ системы (\ref{eq:LSb}) называется устойчивым по Ляпунову, если:
  \begin{gather*}
    (\forall \varepsilon > 0)(\exists \delta > 0)(\forall y(t)) \\
    (\|x(t_0)-y(t_0)\| < \delta \Rightarrow \|x(t)-y(t)\| < \varepsilon)
  \end{gather*}
  в т.ч. при $t \rightarrow \infty$.
\end{importantblock}
\begin{center}
  \small{Определение 1.}
\end{center}
Здесь $y(t)$ - другое решение (\ref{eq:LSb}), которое отличается начальными условиями.
\begin{importantblock}
  Решение $x(t)$ системы (\ref{eq:LSb}) называется асимптотически устойчивым по Ляпунову, если оно устойчиво, и дополнительно выполняется условие:
  \[\lim_{t \to \infty}(x(t) - y(t)) = 0\]
\end{importantblock}
\begin{center}
  \small{Определение 2.}
\end{center}
Аналогичные определения можно ввести для систем разностных уравнений \(x_{n+1} = F(n, x_n)\) с заменой непрерывной переменной $t$ на целую переменную $n$. \\
Для некоторых систем уравнений вывод об устойчивости решения можно сделать, не получая решения в явном виде. Так, например, для линейных дифференциальных уравнений
  с постоянной матрицей выводы об устойчивости можно сделать на основе собственных значений этой матрицы.

\section{Устойчивость решений линейных дифференциальных уравнений с постоянной матрицей}
\setcounter{equation}{0}
\begin{equation}
  \frac{dx}{dt} = Ax+f(t),\, x(0) = x_0
  \label{eq:LRLDLE}
\end{equation}
Решение (\ref{eq:LRLDLE}):
\begin{equation}
  x(t) = e^{At}x_0 + \int_0^t e^{A(t-\tau)} f(\tau)\,d\tau
  \label{eq:LRLDLESol}
\end{equation}
Пусть $y(t)$ - другое решение (\ref{eq:LRLDLE}), отличающееся начальными условиями:
\begin{equation}
  y(t) = e^{At}y_0 + \int_0^t e^{A(t-\tau)} f(\tau)\,d\tau
  \label{eq:LRLDLESol*}
\end{equation}
Вычтем (\ref{eq:LRLDLESol}) из (\ref{eq:LRLDLESol*}):
\begin{gather*}
  x(t) - y(t) = e^{At}(x_0 - y_0) \\
  \mathbf{\varepsilon\delta-language\footnotemark}:\, \|x(t) - y(t)\| \le \|e^{At}\| \times \|x_0 - y_0\|
\end{gather*}
\footnotetext{язык "эпсилон-дельта"}
Для выполнения условия устойчивости элементы матрицы $e^{At}$ при $t \rightarrow \infty$ должны быть ограничены, а для асимптотической устойчивости - 
  стремиться к $0$. \\
Пусть первоначально собственные значения матрицы $A$ различны. Воспользуемся формулой Лагранжа-Сильвестра \footnote{теорема №7 о матричных функциях}:
\begin{gather*}
  f(A) = \sum_{k=1}^N T_k f(\lambda_k) \\
  T_k = \frac{(A-\lambda_1E)\dots(A-\lambda_{k-1}E)(A-\lambda_{k+1}E)\dots(A-\lambda_NE)}
             {(\lambda_k-\lambda_1)\dots(\lambda_k-\lambda_{k-1})(\lambda_k-\lambda_{k+1})\dots(\lambda_k-\lambda_N)}
\end{gather*}
Запишем матричную экспоненту по формуле Лагранжа-Сильвестра:
\[e^{At}=\sum_{k=1}^N T_k e^{\lambda_kt}\]
  \subsection{Достаточное условие для асимптотической устойчивости}
  \begin{align*}
    \lambda_k &= \alpha + i\omega \\
    e^{\lambda_kt} &= e^{\alpha t} \times e^{i\omega t} = e^{\alpha t}(\cos(\omega t)+i\sin(\omega t))
  \end{align*}
  \[\Re(\lambda_k)\footnote{действительная часть} < 0,\, \forall \lambda_k\]
  Если среди собственных значений есть кратные, появляются составляющие в решении\footnote{$\lambda_k^*$ здесь и далее - кратные собств. знач.}:
  \[e^{\lambda_k^*t}P_{S-1}(t)\]
  , где $S$ - кратность корня, а $P_{S-1}$ - полином степени $S-1$.
  
  \subsection{Достаточное условие для устойчивости}
  \[(\Re(\lambda_k) \le 0,\, \forall \lambda_k) \land (\Re(\lambda_k^*) < 0,\, \forall \lambda_k^*)
  \footnote{Все $\Re(\lambda_k)$ меньше либо равны $0$, и среди собственных значений с $\Re(\lambda_k) = 0$ не должно быть кратных.}\]

  \subsection{Необходимое условие для неустойчивости}
  \[(\Re(\lambda_k) > 0,\, \exists \lambda_k) \lor (\Re(\lambda_k^*) = 0,\, \exists \lambda_k^*)
  \footnote{Наличие хотя бы одного собственного значения с $\Re(\lambda_k) > 0$, или кратного собственного значения с $\Re(\lambda_k) = 0$}\]

\section{Устойчивость решений линейных разностных уравнений с постоянной матрицей}
\setcounter{equation}{0}
\begin{equation}
  x_{n+1} = Bx_n + \varphi(n)
  \label{eq:LRLDE}
\end{equation}
Решение (\ref{eq:LRLDE}):
\begin{equation}
  x_n = B^nx_0 + \sum_{k=0}^{n-1}B^k\varphi_{n-1-k}
  \label{eq:LRLDESol}
\end{equation}
Пусть $y(t)$ - другое решение (\ref{eq:LRLDE}), отличающееся начальными условиями:
\begin{equation}
  y_n = B^ny_0 + \sum_{k=0}^{n-1}B^k\varphi_{n-1-k}
  \label{eq:LRLDESol*}
\end{equation}
Вычтем (\ref{eq:LRLDESol}) из (\ref{eq:LRLDESol*}):
\begin{gather*}
  x_n - y_n = B^n(x_0 - y_0) \\
  \mathbf{\varepsilon\delta-language}:\, \|x_n - y_n\| \le \|B^n\| \times \|x_0 - y_0\|
\end{gather*}
Для выполнения условия устойчивости элементы матрицы $B^n$ при $t \rightarrow \infty$ должны быть ограничены, а для асимптотической устойчивости - 
  стремиться к $0$. \\
Пусть первоначально собственные значения матрицы $B$ различны. Воспользуемся формулой Лагранжа-Сильвестра:
\begin{equation*}
  B^n = \sum_{k=1}^N T_k\lambda_k^n
\end{equation*}
Аналогично дифференциальным уравнениям, имеем:
  \subsection{Достаточное условие для асимптотической устойчивости}
  \[|\lambda_k| < 1,\, \forall \lambda_k\]
  Если среди собственных значений есть кратные, появляются составляющие в решении:
  \[(\lambda_k^*)^nP_{s-1}(n)\]
  , где $S$ - кратность корня, а $P_{S-1}$ - полином степени $S-1$.
  
  \subsection{Достаточное условие для устойчивости}
  \[(|\lambda_k| \le 1,\, \forall \lambda_k) \land (|\lambda_k^*| < 1,\, \forall \lambda_k^*)
  \footnote{Все $|\lambda_k|$ меньше либо равны $1$, и среди собственных значений с $|\lambda_k| = 1$ не должно быть кратных.}\]

  \subsection{Необходимое условие для неустойчивости}
  \[(|\lambda_k| > 1,\, \exists \lambda_k) \lor (|\lambda_k^*| = 1,\, \exists \lambda_k^*)
  \footnote{Наличие хотя бы одного собственного значения с $|\lambda_k| > 1$, или кратного собственного значения с $|\lambda_k| = 1$}\]

\begin{mainblock}
  А как связаны между собой понятия устойчивости решения и его ограниченности?
\end{mainblock}
\end{document}

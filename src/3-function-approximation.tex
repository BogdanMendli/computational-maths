\documentclass[a4paper,11pt]{article}

\usepackage{wrapfig}
\usepackage{amsmath}
\usepackage{pgfplots}
%Russian-specific packages
%--------------------------------------
\usepackage[T2A]{fontenc}
\usepackage[utf8]{inputenc}
\usepackage[russian, english]{babel}
%--------------------------------------

\title{03. Введение в аппроксимацию функций. Основы интерполирования.}
\author{Андрей Бареков \and По лекциям Устинова С.М.}
\date{\today}

\begin{document}
\maketitle
\newpage

\section{Введение в аппроксимацию}
Исходная функция чаще всего записывается в следующем виде:
\begin{enumerate}
  \item Аналитически
  \item Графически
  \item Таблично
  \item Алгоритмически
\end{enumerate}
Аппроксимирующая функция должна быть достаточно простой с точки зрения решаемой задачи. \\
Для сравнения различных аппроксимирующих функций вводится критерий близости: \\
Пусть $f(x)$ - исходная функция, $g(x)$ - её аппроксимация.
\begin{enumerate}
  \item \(\delta = \max_{x \in [a,b]} |f(x) - g(x)| \to \min\) - минимаксный критерий.
  \item \(\rho^2 = \int_{a}^{b} (f(x) - g(x))^2 \to \min\) - среднеквадратичный критерий. \\
        Если функция задана таблично, то есть дискретный аналог среднеквадратичного критерия: \\
        \(\rho^2 = \sum_{k=1}^{n} (f(x_k) - g(x_k))^2 \to \min\)
\end{enumerate}

  \subsection{Сравнение критериев}
  \begin{minipage}{1\linewidth}
    \begin{wrapfigure}{l}{8cm}
    \begin{tikzpicture}
      \begin{axis} [
        xlabel = {}, ylabel = {},
        xmin = 0, ymin = 0,
        xtick = {}, ytick = {},
        no markers
      ]
        \addplot+[smooth] [
        color = green,
        thick,
        domain = 0:7,
        samples = 300
      ] coordinates { (0,2)(1,2.9)(1.5,3.1)(2,3)(3,2.6)(4,2.6)(5,3.1)(6,4) };
        \addlegendentry{$f(x)$}

        \addplot+[smooth] [
        color = blue,
        ultra thin,
        domain = 0:7,
        samples = 300
      ] coordinates { (0,2)(0.5,2.5)(1,2.9)(1.25,4)(1.5,6)(1.75,4)(2,3)(2.5,2.8)(3,2.6)(4,2.6)(5,3.1)(6,4) };
        \addlegendentry{$g_1(x)$}

        \addplot+[smooth] [
        color = red,
        ultra thin,
        domain = 0:7,
        samples = 300
      ] coordinates { (0,1.6)(1,2.5)(1.5,2.7)(2,2.6)(3,2.2)(4,2.2)(5,2.7)(6,3.6) };
        \addlegendentry{$g_2(x)$}

      \end{axis}
    \end{tikzpicture}
    \end{wrapfigure}
    Функция $g_1(x)$ лучше аппроксимирует по критерию 2, а $g_2(x)$ - по критерию 1. \\
    На практике лучше та аппроксимация, которая нужна для конкретной задачи.
  \end{minipage}
\newpage

\section{Основы интерполирования функций}
\begin{minipage}{1\linewidth}
  \begin{equation}
    Q_m(x) = \sum_{k=0}^{m} a_k \phi_k(x)
    \label{eq:GenPol}
  \end{equation}
  \begin{wraptable}{l}{2.5cm}
    \begin{tabular}{ c|c }
      $x$ & $f(x)$ \\
      \hline
      $x_0$ & $f(x_0)$ \\
      $x_1$ & $f(x_1)$ \\
      $\cdots$ & $\cdots$ \\
      $x_m$ & $f(x_m)$
    \end{tabular}
  \end{wraptable}

  Аппроксимирующая функция \ref{eq:GenPol} - обобщённый многочлен. \\
  Потребуем, чтобы во всех узлах таблицы аппроксимирующая и исходная функции совпадали.
  \begin{equation}
    Q_m(x_i) = f(x_i), \, i = 0, 1, \dots, m
    \label{eq:PolSys}
  \end{equation}
  Если эти условия выполняются, то $Q_m(x)$ - интерполяционный полином.  
\vspace{2mm}
\end{minipage}
\newline
Система \ref{eq:PolSys} - это линейная система из $m+1$ уравнения относительно $m+1$ неизвестных $a_k$. Если определитель этой системы не равен $0$, то задача имеет единственное решение. \\
Самая популярная интерполяция - интерполяция полиномом - \(\phi_k(x) = x^k\).
При такой интерполяции определитель системы \ref{eq:PolSys} приобретает следующий вид:
\begin{equation}
  \begin{vmatrix}
    1 & x_0 & x_0^2 & \cdots & x_0^m \\
    1 & x_1 & x_1^2 & \cdots & x_1^m \\
    \vdots & \vdots & \vdots & \ddots & \vdots \\
    1 & x_m & x_m^2 & \cdots & x_m^m
  \end{vmatrix}
  \label{eq:VandDet}
\end{equation}
Определитель \ref{eq:VandDet} - определитель Вандермонда, не равный $0$. \\
По $n$ точкам однозначно строится интерполяционный полином $n-1$й степени. \\
Построить интерполяционный полином можно и в явном виде, не решая систему \ref{eq:PolSys}.
\begin{gather*}
  \omega(x) = (x - x_0)(x - x_1)\dots(x - x_m) \\
  \omega_k(x) = \frac{\omega(x)}{x - x_k} = (x - x_0)\dots(x - x_{k-1})(x - x_{k+1})\dots(x - x_m)
\end{gather*}

\begin{equation}
  Q_m(x) = \sum_{k=0}{m} \frac{\omega_k(x)}{\omega_k(x_k)} f(x_k)
  \label{eq:LgPol}
\end{equation}
Уравнение \ref{eq:LgPol} - интерполяционный полином (в форме) Лагранжа. \\
По построению очевидно, что $Q_m(x)$ - полином степени $m$.
\[Q_m(x_i) = \frac{\omega_i(x_i)}{\omega_i(x_i)} f(x_i) = f(x_i), \, i=0,1,\dots,m\]
\(\omega_k(x_i) = 0\), если \(k \ne i\)

\end{document}
